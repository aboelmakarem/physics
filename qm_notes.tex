\documentclass[8pt,t,mathserif,aspectratio=169]{beamer}

%%% Define Packages.  At a minimum, you will need graphicx.  
%%% The "ams" packages and mchem are tremendously useful, but are probably not installed by default.  
\usepackage{graphicx}			
\graphicspath{{Figures/}}
\usepackage{amsmath}
\usepackage{amssymb, amsthm}
\usepackage{xcolor}
\usepackage{parskip}
\usepackage{scrextend}
\usepackage{physics}

%%%%%%%%%%%%%%%%%%%%%%%%%%%%%%%%%%%%%%
%%%% Setup for footers, headers, etc....  %%%%%%%%%%%%%%%%%
%%%% Don't change anything here          %%%%%%%%%%%%%%%%%
%%%%%%%%%%%%%%%%%%%%%%%%%%%%%%%%%%%%%%
\setlength{\parskip}{5pt} 
\usepackage{helvet}
\renewcommand{\familydefault}{\sfdefault}
\beamertemplatenavigationsymbolsempty
\newcommand\epinfo{QM Notes by Ahmed Hussein}
\definecolor{jhublue}{rgb}{0.133,0.184,0.337}
\definecolor{jhutext1}{rgb}{0.0314,0.294,0.510}
\setbeamercolor{footlinecolor}{fg=white,bg=jhublue}
\makeatother
\setbeamertemplate{footline}{%
  \leavevmode%
  \hbox{%
  \begin{beamercolorbox}[wd=1.0\paperwidth,ht=5mm,dp=1mm,leftskip=.3cm,rightskip=.3cm]{footlinecolor} \epinfo \hfill  \insertframenumber{} \hfill
    \vskip0pt plus.5fill
  \end{beamercolorbox}}%
  \vskip0pt%
}
\makeatletter

\makeatletter
\newenvironment{nofootline}
{
    \setbeamertemplate{footline}{}
    \addtobeamertemplate{frametitle}{\vspace*{-0.9\baselineskip}}{}
}{}
\makeatother

%%%%%%%%%%%%%%%%%%%%%%%%%%%%%%%%%%%%%%
%%%% End of front matter.                       %%%%%%%%%%%%%%%%%
%%%% Your slides start here.                   %%%%%%%%%%%%%%%%%
%%%%%%%%%%%%%%%%%%%%%%%%%%%%%%%%%%%%%%

%%%% Standard title slide  %%%%%%%%%%%%%%%%%%%%%%%%
%%%%%%%%%%%%%%%%%%%%%%%%%%%%%%%%%%%%%%
%%%%%%%%%%%%%%%%%%%%%%%%%%%%%%%%%%%%%%
%%%%%%%%%%%%%%%%%%%%%%%%%%%%%%%%%%%%%%

\begin{document}

% \changefontsizes{7.5pt}
\begin{nofootline}
\begin{frame}
  \vspace{18mm}
  \begin{center}
  \Huge
  \textcolor{jhutext1}{Quantum Mechanics Notes} \\
  \LARGE
  \vspace{2mm}
  Ahmed Hussein \\
  \vspace{7mm}
  \end{center}
\end{frame}
\end{nofootline}
% \changefontsizes{9.0pt}

%%%%%%%%%%%%%%%%%%%%%%%%%%%%%%%%%%%%%
\begin{frame}
  \frametitle{Postulates of Quantum Mechanics}
  \vspace{1mm}
  The following statements are postulates of quantum mechanics and can not be derived from anything more fundamental. They can only be, and have extensively been, verified by experiments
  \begin{itemize}
    \item The state of a quantum mechanical system is represented by a ket $\ket{\psi}$ in a Hilbert space
    \item All of the system's dynamical variables are represented by Hermitian operators
    \begin{itemize}
      \item Dynamical variables can be positions, linear momenta, angular momenta, energies, etc...
    \end{itemize}
    \item The measurement of a variable represented by an operator $A$ for a system in state $\ket{\psi}$ results in the observation of one of $A$'s eigenvalues
    \item The probability of getting a measurement $a$ for a dynamical variable $A$ is proportional to the square of the inner product of the system's ket $\ket{\psi}$ and $A$'s eigenket $\ket{a}$ corresponding to its eigenvalue $a$
    \item Upon performing a measurement, the system's state changes to the eigenket corresponding to the measurement outcome eigenvalue (wavefunction collapse)
    \item The system evolves in time according to Schrodinger's equation $i \hbar {d \over dt} \ket{\psi} = H \ket{\psi}$
    \begin{itemize}
      \item $H$ is an operator called the Hamiltonian of the system and it depends on the nature of the system and its interactions with its environment
    \end{itemize}
  \end{itemize}
\end{frame}

\begin{frame}
  \frametitle{Quantum Mechanical Systems}
  \vspace{1mm}
  A Quantum Mechanical (QM) system consists of a number $N$ of particles. Each particle has up to $4$ degrees of freedom (DoF)
  \begin{itemize}
    \item Up to $3$ position DoFs for the $3$ spatial dimensons and up to $1$ spin DoF.
    \item All DoFs will be assumed to take values in infinite and continuous ranges until proven otherwise
  \end{itemize}
  Since the DoFs are dynamical variables themselves, there are Hermitian operators that represent them
  \begin{itemize}
    \item The operators $X_i, i = 1,2,3$ represent the $x,y$ and $z$ position DoFs of a particle
    \item The operator $S$ represents the spin DoF of the particle
    \item Continuous/discrete DoFs have continuous/discrete eigenkets
    \end{itemize}
  % An N-particle system in $3D$ has $3N$ positions represented by operators $X_i, i = 1 \to 3N$ and $N$ spins represented by operators $S_i, i = 1 \to N$
  The Hilbert space containing the kets describing one-particle systems is spanned by kets $\ket{x_1 x_2 x_3 s}$ each corresponding to a particle position $(x_1,x_2,x_3)$ and spin $s$
  \begin{itemize}
    \item In general, the system's state ket $\ket{\psi}$ is a linear combination of the kets $\ket{x_1 x_2 x_3 s}$
    \item The abstract ket $\ket{x_1 x_2 x_3 s}$ will be denoted by $\ket{x,s}$
    \item The following development will be for a single particle systems, multi-particle systems will be handled later on
    % \item The Hilbert space containing the kets describing N-particle systems is spanned by kets $\ket{x_1 x_2 ... x_{3N} s_1 s_2 ... s_N}$ each corresponding to the N particles being at positions $(x_1,x_{N+1},x_{2N+1}), (x_2,x_{N+2},x_{2N+2}), ..., (x_N,x_{2N},x_{3N})$ and having spins $s_1,s_2,...,s_N$
  \end{itemize}
\end{frame}

\begin{frame}
  \frametitle{Kets of QM Systems}
  \vspace{1mm}
  Any system ket $\ket{\psi}$ can be decomposed in the position-spin basis $\ket{x,s}$ where $x$ stands for all position variables and $s$ stands for the spin variable
  \begin{itemize}
    \item $\ket{\psi} = \int_x \int_s \psi(x,s) \ket{x,s} dx ds$
    \item Here, $\int_x .. dx_i$ and $\int_s .. ds$ are integrations over all position/spin DoFs, let $\int_{\Omega} ... d\Omega = \int_x \int_s ... dx ds$
    \item Integrals become sums if any of the DoFs turn out to be discrete
    \item Hence $\ket{\psi} = \int_{\Omega} \psi(x,s) \ket{x,s} d\Omega$ where $x$ is a point in the $3$-dimensional position space and $s$ is a point in the $1$-dimensional spin space
    \item $\psi(x,s)$ is called the system's wave function. It is a function in all of the system's degrees of freedom
    \item $\psi(x,s)$ is the projection of $\ket{\psi}$ on $\ket{x,s}$: $\braket{x,s}{\psi} = \int_{\Omega} \psi(x',s') \braket{x,s}{x',s'} d\Omega = \psi(x,s)$
  \end{itemize}
  The operators $X_i$ and $S$ will be chosen such that any ket $\ket{x,s}$ is a simultaneous eigenket for all of them
  \begin{itemize}
    \item This is required in order to allow the system's particles to have defined positions and spins simultaneously
    \item Otherwise, particles in certain positions will have a distribution of spins and vice-versa which is inconsistent with positions and spins being independent DoFs
    \item Hence, the commutators $[X_i,X_j] = [X_i,S] = 0$ for all $i$ and $j$
  \end{itemize}
\end{frame}

\begin{frame}
  \frametitle{Position and Spin Operators}
  \vspace{1mm}
  For the position operator $X_k$ to have the ket $\ket{x,s}$ as an eigenket, $X_k \ket{x,s} = x_k \ket{x,s}$
  \begin{itemize}
    \item $x_k$ is the value of the $k^{th}$ spatial coordinate when the system is in the state $\ket{x,s}$
    \item Consequently, $\braket{x,s}{X_k \psi} = \bra{x,s X_k} \ket{\psi} = x_k \braket{x,s}{\psi} = x_k \psi(x,s)$
    \item Hence, the position operator $X_k$ acts on a ket $\ket{\psi}$ by multiplying it by the $k^{th}$ coordinate: $X_k \ket{\psi} = x_k \ket{\psi}$
    \item This is consistent with the commutation requirement $[X_i,X_j] = 0$ since $[X_i,X_j] = x_i x_j - x_j x_i = 0$
    \item The position operators do not depend on the spin DoF and they do not affect it
  \end{itemize}
  Similarly, for the spin operator $S$ to have the ket $\ket{x,s}$ as an eigenket, $S \ket{x,s} = s \ket{x,s}$
  \begin{itemize}
    \item $s$ is the value of the spin when the system is in the state $\ket{x,s}$
    \item Consequently, $\braket{x,s}{S \psi} = \bra{x,s S} \ket{\psi} = s \braket{x,s}{\psi} = s \psi(x,s)$
    \item Hence, the spin operator $S$ acts on a ket $\ket{\psi}$ by multiplying it by the spin: $S \ket{\psi} = s \ket{\psi}$
    \item The spin operators do not depend on the position DoFs and they do not affect them
  \end{itemize}
  Since all position and spin eigenvalues are real, the position-spin commutation requirement is satisfied
  \begin{itemize}
    \item $[X_i,S] = 0$ since $[X_i,S] = x_i s - s x_i = 0$
  \end{itemize}
\end{frame}

\begin{frame}
  \frametitle{Spin}
  \vspace{1mm}
  Spin DoFs do not correspond to any physical spin for any particles and they do not have a classical analog
  \begin{itemize}
    \item Experimentally, particles in QM systems exhibit an intrinsic angular momentum that is not associated with their orbital angular momenta
    \item It is an angular momentum because spinning particles interact with magnetic fields the same way as particles with orbital angular momenta 
    \item The intrinsic angular momentum is called spin and it is a particle property like mass and charge
    \item Unlike mass and charge however, a particle's spin can change over time as in interacts with other particles and external electromagnetic fields
    \item It will be shown later that spin DoF takes one of $M$ discrete values. Let these values be $s_1,s_2,...,s_M$
    \item The number $M$ of allowable spin DoF values and their values $s_1,s_2,...,s_M$ depend on the nature of the system's particles
  \end{itemize}
  Magnetic fields are vector fields and have a preferred direction locally, hence, the particle's spin needs to have a directional preference as well in order to properly model its interaction with magnetic fields. 
  \begin{itemize}
    \item For each particle, there is a unit vector in $3D$ with components $(n_x,n_y,n_z)$ called the spin axis
    \item $n_x^2 + n_y^2 + n_z^2 = 1$
    \item Different particles have, in general, different spin axes
  \end{itemize}
\end{frame}

\begin{frame}
  \frametitle{Spin Axis}
  \vspace{1mm}
  Let the spin operator for a given spin axis $\hat{n}$ for the particle be $S^n$
  \begin{itemize}
    \item If the system's ket $\psi$ is an eigenket of $S^n$, we say that the spin axis of the particle is $\hat{n}$
    % \item The forms of $S_x, S_y$ and $S_z$ depend on the particle's nature and will be determinated later
    % \item Form the operator $S_n = n.S = n_x S_x + n_y S_y + n_z S_z$, this is the spin operator in the $\hat{n}$ direction
    \item If not, then particle spins about some other axis which can be obtained by varying $n_x,n_y$ and $n_z$ until the $\ket{\psi}$ becomes an eigenket of $S^n$ 
    % \item For multiple-particle systems, the spin operators for different particles commute regardless of the spin axes since the spin DoFs are independent $[S^{n_i}_i,S^{n_j}_j] = 0$
    % \item For spin axes in the $x,y$ and $z$ directions, the $i^{th}$ particle spin operators are $S^x_i,S^y_i$ and $S^z_i$ respectively
    \item For any $\hat{n}$, $S^n$ has $M$ eigenvalues equal to $s_1,s_2,...,s_M$, the $M$ spin DoF allowable values for the particle
    \item As $\hat{n}$ changes, the eigenkets corresponding to the $M$ eigenvalues change 
    \item A coordinate system rotation causes the spin axis components $n_i$ to change (vector rotation)
    \item This is not the case for a coordinate system translation or a time shift where the spin axis remains the same
    \item Spatial rotations with an orthonormal rotation matrix $R$ change the spin axis $n_i$ to $n'_i = R_{ij} n_j$
    \item Operating with an operator $S^n$ on a ket $\ket{\psi}$ of a particle that does not have $\hat{n}$ as its spin axis results in the multiplication of the eigenvalues of $S^n$ with the components of $\ket{\psi}$ in $S^n$ eigenkets directions as usual
  \end{itemize}
\end{frame}


% \begin{frame}
%   \frametitle{Spinors}
%   \vspace{1mm}
%   The position operator eigenkets $\ket{x_i}$ are members of a position Hilbert space $H_x$. Similarly, the spin operator eigenkets $\ket{s_j}$ are members of a spin Hilbert space $H_s$
%   \begin{itemize}
%     \item The product space $H = H_x \otimes H_s$ is spanned by basis kets that are the tensor products of kets $\ket{x_i}$ and $\ket{s_j}$
%     \item The system kets $\ket{\psi}$ are members of the product Hilbert space $H$
%     \item Hence, $\ket{x_i,s_j} = \ket{x_i} \ket{s_j} \implies \braket{x_i,s_j}{x_l,s_m} = \braket{x_i}{x_l} \braket{s_j}{s_m} = \delta_{jm} \braket{x_i}{x_l}$
%   \end{itemize}
%   Any ket $\ket{\psi}$ can be written as a linear combination of the position-spin eigenkets
%   \begin{itemize}
%     \item $\ket{\psi} = \Sigma^{M}_{j = 1} \int_{\Omega} \psi(x_i,s_j) \ket{x_i,s_j} d\Omega = \Sigma^{M}_{j = 1} \int_{\Omega} \psi(x_i,s_j) \ket{x_i}\ket{s_j} d\Omega$
%     \item Then, $\braket{s_k}{\psi} = \Sigma^{M}_{j = 1} \int_{\Omega} \psi(x_i,s_j) \braket{s_k}{s_j}\ket{x_i} d\Omega = \Sigma^{M}_{j = 1} \int_{\Omega} \psi(x_i,s_j) \delta_{kj}\ket{x_i} d\Omega = \int_{\Omega} \psi(x_i,s_k) \ket{x_i} d\Omega$
%     \item Let $\ket{\psi_k} = \int_{\Omega} \psi(x_i,s_k) \ket{x_i} d\Omega \implies \ket{\psi_k}$ is a ket in $H_x$ and $\braket{x_i}{\psi_k} = \psi_k(x_i)$
%     \item Consequently, $\ket{\psi} = \ket{\psi_1} \ket{s_1} + \ket{\psi_2} \ket{s_2} + ... + \ket{\psi_M} \ket{s_M} = \ket{\psi_j} \ket{s_j}$
%     \item $\ket{\psi}$ is the sum of outer products of $H_x$ kets $\ket{\psi_j}$ and $H_s$ basis kets $\ket{s_j}$
%   \end{itemize}
%   Hence, $\ket{\psi}$ (a ket in $H$) can be thought of as a vector-like object with components $\ket{\psi_k}$ (kets in $H_x)$
%   \begin{itemize}
%     \item The vector-like object is called a spinor and it is a member of $H$ with components in $H_x$
%     \item Operators acting on $H_x$ kets operate on spinor components while operators acting on $H_s$ kets operate on the spinor itself (by forming linear combinations of its components)
%   \end{itemize}
% \end{frame}

% \begin{frame}
%   \frametitle{Mathematical Framework of Quantum Mechanics}
%   \vspace{1mm}
%   The state of a QM system is represented by a time-varying spinor $\ket{\psi(t)}$ in a Hilbert space $H$
%   \begin{itemize}
%     \item The number of the spinor components $M$ depends on the nature of the system's particles
%   \end{itemize}  
%   Operators on $H$ kets are either tensor produts of operators on $H_x$ and $H_s$ or linear combinations of such tensor products
%   \begin{itemize}
%     \item A tensor product opetator $A$ on $H$ is one where $A = A_x \otimes A_s$ with $A_x$ on $H_x$ and $A_s$ on $H_s$
%     \item $A \ket{\psi} = (A_x \otimes A_s) \ket{\psi} = A_s \begin{pmatrix} A_x \ket{\psi_1} \\ A_x \ket{\psi_2} \\ \vdots \\ A_x \ket{\psi_M} \end{pmatrix} = A_x (A_s \begin{pmatrix} \ket{\psi_1} \\ \ket{\psi_2} \\ \vdots \\ \ket{\psi_M} \end{pmatrix})$ where $A_s$ is an $M \times M$ matrix
%     \item For a linear combination of the tensor products, the previous operation holds term by term. The resulting ket for each term is multiplied by the term coefficient and all kets are summed to get the final ket
%   \end{itemize}
%   The spinor encodes the entire state of the QM system. Anything that can be known about the system in the present or the future can be obtained from the system's spinor
%   \begin{itemize}
%     \item Spinors are used because the spin DoFs are discrete, had they been continuous, we could have used a regular function with extra independent variables for spin DoFs 
%   \end{itemize}
% \end{frame}

\begin{frame}
  \frametitle{Unitary Transformations and Their Generators}
  \vspace{1mm}
  Take any Hermitian operator $K$ and construct the transformation $U(\epsilon) = I + i \epsilon K$
  \begin{itemize}
    \item $\epsilon$ is a real number and $I$ is the identity operator, $U(\epsilon)$ is parametrized by $\epsilon$
    \item Since $K = K^{\dagger} \implies U^{\dagger}(\epsilon) = I - i \epsilon K \implies U^{\dagger} U = I - \epsilon^2 K^2$
    \item For infinitesimal $\epsilon$: $U^{\dagger} U = I \implies U$ is unitary
    \item This is only true because $K$ is Hermitian
  \end{itemize}
  $U(\epsilon)$ is an infinitesimal unitary transformation that transforms all kets and operators and $K$ is the generator of this transformation
  \begin{itemize}
    \item For finite transformations: $U(s) = lim_{N \to \infty} (U({s \over N}))^{N} = lim_{N \to \infty} (I + i {s \over N} K)^{N} = \exp(i s K)$
    \item $U^{\dagger}(s) = \exp(-isK) \implies U^{\dagger}(s) U(s) = I \implies U(s)$ is unitary
  \end{itemize}
  For any operator $A$ with eigenkets $\ket{a}$, a ket $\ket{\psi}$ and a unitary transformation $U(s)$:
  \begin{itemize}
    \item The ket $\ket{\psi'} = U(s) \ket{\psi}$ is the transformation of $\ket{\psi}$ under the action of $U(s)$
    \item There exists an operator $A'$ such that the eigenkets $\ket{a}$ of $A$ transform to kets $\ket{a'} = U(s) \ket{a}$ that are the eigenkets of $A'$ with the same eigenvalues: $A \ket{a_i} = a_i \ket{a_i} \implies A' \ket{a_i'} = a_i \ket{a_i'}$
  \end{itemize}
\end{frame}

\begin{frame}
  \frametitle{Symmetry Transformations}
  \vspace{1mm}
  For an arbitrary operator $A$ with eigenkets $\ket{a}$, if a unitary transformation $U(s)$ transforms the kets $\ket{\psi}$ such that
  \begin{itemize}
    \item If $\ket{\psi} = \Sigma_i c_i \ket{a_i}$ then $\ket{\psi'} = \Sigma_i c_i' \ket{a_i'}$ where $|c_i| = |c_i'| \implies |\braket{\psi'}{a_i'}| = |\braket{\psi}{a_i}|$ 
    \item That is, the transformation preserves the probability of observing an eigenvalue $a_i$ upon measurement
  \end{itemize}
  Then $U(s)$ is a symmetry transformation because all system measurements performed before and after the transformation yield outcomes with the same probabilities

  For this to hold, $U(s)$ has to be a unitary transformation
  \begin{itemize}
    \item $\ket{a'} = U \ket{a}$ and $\ket{\psi'} = U \ket{\psi} \implies \braket{\psi'}{a'} = \bra{\psi U^{\dagger}}\ket{U \psi} = \braket{\psi}{a}$
    \item $A' \ket{a_i'} = a_i \ket{a_i'} = a_i U \ket{a_i} \implies A' U \ket{a_i} = a_i U \ket{a_i} \implies U^{\dagger} A' U \ket{a_i} = a_i \ket{a_i}$
    \item $U^{\dagger} A' U \ket{a_i} = a_i \ket{a_i}$ and $A \ket{a_i} = a_i \ket{a_i}$ for all $i \implies A = U^{\dagger} A' U \implies A' = U A U^{\dagger}$ 
  \end{itemize}

  In summary, any Hermitian operator $K$ can generate a symmetry transformation $U(s) = \exp(i s K)$
  \begin{itemize}
    \item Any ket $\ket{\psi}$ transforms to $\ket{\psi'} = U(s) \ket{\psi}$
    \item There exists an operator $A' = U A U^{\dagger}$ such that the eigenkets $\ket{a_i}$ of $A$ transform to eigenkets $\ket{a'_i}$ of $A'$ with the same eigenvalues $a_i$
    \item The expansion coefficients of any ket $\ket{\psi}$ in any basis $\ket{a}$ are preserved: $\braket{\psi'}{a'} = \braket{\psi}{a}$
  \end{itemize}
\end{frame}

\begin{frame}
  \frametitle{Coordinate System Transformations}
  \vspace{1mm}
  A transformation of the space-time coordinates (STC) used to describe a physical system can not change the state of the system or the laws governing its evolution. Hence, the effect of an STC transformation can only be modeled by a symmetry transformation

  Nature is invariant under the following STC transformations, called invariance transformations (IT)
  \begin{itemize}
    \item A displacement in any spatial direction ($3$ ITs with generators $-P_1$, $-P_2$ and $-P_3$)
    \item A rotation about any spatial axis ($3$ ITs with generators $-J_1$, $-J_2$ and $-J_3$)
    \item A motion with uniform velocity in any spatial direction ($3$ ITs with generators $G_1$, $G_2$ and $G_3$)
    \item A shift (translation) in time ($1$ IT with generator $H$)
    \item Any combination of the above transformations in any order
  \end{itemize}
  For a transformation $\tau$, the space-time coordinates transform according to $x'_i = R_{ij} x_j + d_i + v_i t, t' = t + \delta$ 
  \begin{itemize}
    \item $R_{ij}$ is an orthonormal rotation matrix, $d_i$ is a displacement vector, $v_i$ is a velocity vector and $\delta$ is a time shift
    \item For transformations $\tau_1(R_1,d_1,v_1,\delta_1)$ and $\tau_2(R_2,d_2,v_2,\delta_2)$, the product transformation $\tau_{12} = \tau_2 \tau_1$
    \item $\tau_{12} \implies x' = R_2 R_1 x + R_2 d_1 + R_2 v_1 t + d_2 + v_2 (t + \delta_1)$, $t' = t + \delta_1 + \delta_2$
    \item $\tau_{12}$ rotation, displacement, velocity and time shift are $R_2 R_1$, $R_2 d_1 + d_2 + v_2 \delta_1$, $R_2 v_1 + v_2$ and $\delta_2 + \delta_1$
  \end{itemize}

\end{frame}

\begin{frame}
  \frametitle{Coordinate System Transformations II}
  \vspace{1mm}
  For every transformation $\tau$ transforming one STC to another, there exists a unitary transformation $U(\tau)$ that transforms the kets $\ket{\psi}$ and operators $A$ in one STC to kets $\ket{\psi'}$ and operators $A'$ in the other
  \begin{itemize}
    \item $\ket{\psi'} = U(\tau) \ket{\psi}$ and $A' = U(\tau) A U^{\dagger}(\tau)$
    \item Since $U^{-1}(\tau) = U^{\dagger}(\tau) \implies U(\tau^{-1}) = U^{-1}(\tau) = U^{\dagger}(\tau)$
  \end{itemize}
  When using unitary transformations to transform kets, the transformed ket can differ from the original ket by a phase factor $\exp(i \delta), \delta \in R$ since it this does not change the kets 
  \begin{itemize}
    \item $\delta$ is independent of $A$ and $\psi$ for $U(\tau)$ to be linear
  \end{itemize}
  Upon applying an STC transformation $\tau$, the system ket $\ket{\psi}$ transform to ket $\ket{\psi'} = U(\tau) \ket{\psi}$
  \begin{itemize}
    \item The kets $\ket{x,s}$ transform to $\ket{x',s'}$ where $x' = \tau_x(x)$ and $s' = \tau_s(x)$
    \item To preserve the system state: $\braket{x',s'}{\psi'} = \braket{x,s}{\psi}$ for all $x$ and $s$ (preserve all measurement probabilities)
    \item $\ket{\psi'} = U(\tau) \ket{\psi} \implies \braket{x',s'}{\psi'} = \braket{x',s'}{U(\tau) \psi} \implies \psi'(x',s') = [U(\tau) \psi](x',s')$
    \item But $\psi'(x',s') = \braket{x',s'}{\psi'} = \braket{x,s}{\psi} = \psi(x,s) = \psi'(\tau_x(x),\tau_s(x)) = \psi(\tau^{-1}_x(x'),\tau^{-1}_s(x'))$ 
    \item Then $\ket{\psi'} = U(\tau) \ket{\psi} \implies \psi'(x'_i,s'_j) = [U(\tau) \psi](x'_i,s'_j) = \psi(\tau^{-1}_x(x'_i),\tau^{-1}_s(x'_i))$
    \item Hence, the kets transform in a direction opposite to the coordinate system transformation to preserve their values
  \end{itemize}
\end{frame}

\begin{frame}
  \frametitle{Coordinate System Transformations III}
  \vspace{1mm}
  The transformation $\tau$ depends only on the position $x$ variables of the wave function and not on the spin variables $s$ because every point in space-time is transformed by $\tau$ while the particles' spins are not

  Let $\psi_s(x)$ denote the system's wave function $\psi(x,s)$ evaluated at the spin $s$. The application of an STC transformation changes the system kets in two ways
  \begin{itemize}
    \item The functions $\psi_s(x)$ transform such that their values at one space-time point after transformation are reassigned to the values they had to the same space-time point before transformation $\psi'_s(x') = [U(\tau) \psi_s](x) = \psi_s(\tau^{-1}(x))$
    \item This is required to preseve the particle's position DoFs and all dynamical variables depending on them
    \item Otherwise, a measurement of a dynamical variable on a particle in spin configuration $s$ would yield different outcomes after transformation since the transformation of the function $\psi_s(x)$ does not preserve its values
    \item The spins also get reassigned so that $\psi_s$ becomes $\psi_{s'}$
    \item This is required to preserve the particle's spin DoFs after the transformation 
    \item Otherwise, a particle in an eigenket state of a spin operator $S^n$ in the $\hat{n}$ direction would not necessarily remain and eigenket for the operator $S_{n'}$ in the transformed direction $\hat{n}' = \tau(\hat{n})$
  \end{itemize}
\end{frame}

\begin{frame}
  \frametitle{Coordinate System Transformations IV}
  \vspace{1mm}
  Both position and spin reassignments are linear combinations of the position and spin values before transformation with combination coefficients depending on $\tau$
  \begin{itemize}
    \item Prepare a particle in position $x$ and with spin $s$ and apply the STC transformation $\tau$ to its system
    \item Prepare another particle in position $x$ and with spin $s'$ and apply $\tau$ to its system
    \item Prepare a third particle in position $x'$ and with spin $s$ and apply $\tau$ to its system
    \item In all cases, the transformation should not change the particle's position or spin in the new STC system
    \item Hence, upon transformation, the position reassignments are spin-independent and spin reassignments are position-independent
    \item This suggests that $U(\tau)$ can be broken down into $U_x(\tau)$ and $U_s(\tau) : U(\tau) = U_x(\tau) U_s(\tau) = U_s(\tau) U_x(\tau)$
    \item Here, $U_x(\tau)$ transforms $\psi_s(x)$ to $\psi_s(x')$ while $U_s(\tau)$ transforms $\psi_s(x)$ to $\psi_{s'}(x)$ 
  \end{itemize}
  Hence, the most general effect of an STC transformation on a ket $\ket{\psi}$ is to take it to $\ket{\psi'}$ such that
  \begin{itemize}
    \item $\ket{\psi'} = U_x(\tau) U_s(\tau) \ket{\psi} = U_s(\tau) U_x(\tau) \ket{\psi}$, $U_x(\tau)$ and $U_s(\tau)$ commute because they act on $\ket{\psi}$ DoFs independently 
    \item $\psi'_s(x') = [U_x(\tau) \psi_s](x') = \psi_s(\tau^{-1}(x))$
    \item $U_s(\tau)$ forms linear combinations of $\psi_s$ functions but does not change each of them independently
    % \item $U_x(\tau)$ is a continuous transformation because position DoFs are continuous (a fact of nature)
    % \item The continuity/discreteness nature of spin DoFs dictate the nature of $U_s(\tau)$ transformation. If it is discrete, it will be an (an $M \times M$ matrix)
  \end{itemize}
\end{frame}

\begin{frame}
  \frametitle{Coordinate System Transformations V}
  \vspace{1mm}
  For an STC transformations $\tau(\alpha)$ parametrized by a single continuous parameter $\alpha$, $U_x(\tau) = U_x(\alpha)$ and $U_s(\tau) = U_s(\alpha)$
  \begin{itemize}
    \item Multiple-parameters transformations are the compositions of multiple single-parameter transformation
    \item $U_x(\alpha) = I + \alpha {d U_x \over d\alpha}|_{\alpha = 0} + O(\alpha^2)$ and $U_s(\alpha) = I + \alpha {d U_s \over d\alpha}|_{\alpha = 0} + O(\alpha^2)$
    \item For infinitesimal $\alpha = \epsilon \to 0: U_x(\epsilon) = I + {i \over \hbar} \epsilon K$ and $U_s(\epsilon) = I + {i \over \hbar} \epsilon D$
    \item $K = -i \hbar {d U_x \over d\alpha}|_{\alpha = 0}$ and $D = -i \hbar {d U_s \over d\alpha}|_{\alpha = 0}$ are the Hermitian generators of $U_s(\alpha)$ and $U_s(\alpha)$
    % \item $K$ is a continuous generator while $D$ may be a discrete ($M \times M$ matrix) generator
    \item $\ket{\psi'} = U_x(\epsilon) U_s(\epsilon) \ket{\psi} = (I + {i \over \hbar} \epsilon K)(I + {i \over \hbar} \epsilon T) \ket{\psi} = (I + {i \over \hbar} \epsilon (K + D)) \ket{\psi} = (I + {i \over \hbar} \epsilon V) \ket{\psi}$
    \item $V = K + D$ is the generator of the overall infinitesimal transformation
    \item For finite $\alpha$: $U_x(\alpha) = lim_{N \to \infty} U_x({\alpha \over N})^{N} = lim_{N \to \infty} [I + {i \over \hbar} {\alpha \over N} K]^{N} = \exp({i \over \hbar} \alpha K)$
    \item Similarly, $U_s(\alpha) = \exp({i \over \hbar} \alpha D)$
    \item The overall finite transformation: $U_x(\alpha) U_s(\alpha) = \exp({i \over \hbar} \alpha K) \exp({i \over \hbar} \alpha D) = \exp({i \over \hbar} \alpha D) \exp({i \over \hbar} \alpha K)$
    \item Finite transformation operators commute the same way the infinitesimal transformation operators commute
  \end{itemize}
\end{frame}

\begin{frame}
  \frametitle{Rotation Transformation}
  \vspace{1mm}
  The rotation matrices $R_i(\theta)$ that rotate points of three-dimensional space about an axis $x_i$ by an angle $\theta$ are
  \begin{itemize}
    \item $R_1(\theta) = \begin{pmatrix} 1 & 0 & 0\\ 0 & \cos{\theta} & -\sin{\theta} \\ 0 & \sin{\theta} & \cos{\theta}\end{pmatrix}$, $R_2(\theta) = \begin{pmatrix} \cos{\theta} & 0 & \sin{\theta}\\ 0 & 1 & 0 \\ -\sin{\theta} & 0 & \cos{\theta}\end{pmatrix}$ and $R_3(\theta) = \begin{pmatrix} \cos{\theta} & -\sin{\theta} & 0\\ \sin{\theta} & \cos{\theta} &  0\\ 0 & 0 & 1\end{pmatrix}$
  \end{itemize}
  For rotations by an infinitesimal angle $\epsilon$, $\cos{\epsilon} \to 1$, $\sin{\epsilon} \to \epsilon$ and the rotation matrices become
  \begin{itemize}
    \item $T_1(\epsilon) = \begin{pmatrix} 1 & 0 & 0\\ 0 & 1 & -\epsilon \\ 0 & \epsilon & 1\end{pmatrix}$, $T_2(\epsilon) = \begin{pmatrix} 1 & 0 & \epsilon\\ 0 & 1 & 0 \\ -\epsilon & 0 & 1\end{pmatrix}$ and $T_3(\theta) = \begin{pmatrix} 1 & -\epsilon & 0\\ \epsilon & 1 &  0\\ 0 & 0 & 1\end{pmatrix}$
    \item This can be written as $T_{k}(\epsilon) = I + {i \over \hbar} \epsilon M^k$ where $k = 1,2,3$ is the rotation axis and $M^{k}_{ij} = i \hbar \epsilon_{ijk}$
    \item The transformation $T_k(\epsilon) = I + {i \over \hbar} \epsilon M^k$ is the one that infinitesimally rotates space
  \end{itemize}
  For finite-angle rotations about an axis $x_k$ by an angle $\theta$, the total transformation $R_k(\theta)$ can be composed from an infinite product of infinitesimal transformations $T_k(\epsilon)$
  \begin{itemize}
    \item $R_k(\theta) = lim_{n \to \infty} [T_{k}({\theta \over n})]^n = lim_{n \to \infty} [I + {i \over \hbar} {\theta \over n} M^k]^n = \exp({i \over \hbar} \theta M^{k})$
  \end{itemize}
  The inverse of the infinitesimal rotation transformation $T_k(\epsilon) = I + {i \over \hbar} \epsilon M^k$ is $T^{-1}_k(\epsilon) = I - {i \over \hbar} \epsilon M^k$
  \begin{itemize}
    \item $T_k(\epsilon) T^{-1}_k(\epsilon) = (I + {i \over \hbar} \epsilon M^k)(I - {i \over \hbar} \epsilon M^k) = I - {\epsilon^{2} \over \hbar^2} M^k M^k \to I$ as $\epsilon \to 0$
  \end{itemize}
\end{frame}

\begin{frame}
  \frametitle{Space Translation Generator}
  \vspace{1mm}
  An infinitesimal translation $\epsilon$ in space in the $i$ direction changes the space coordinates but does not change spin direction ($S^n = S^{n'}$ for any spin operator in the direction $\hat{n}' = \hat{n}$) $\implies \ket{\psi'} = U_{xi}(\epsilon) U_{si}(\epsilon) \ket{\psi}$
  \begin{itemize}
    \item The transformed kets $\ket{\psi'} = U_{xi}(\epsilon) U_{si}(\epsilon) \ket{\psi} = U_{xi}(\epsilon) I \ket{\psi} \implies D_i = 0$
    \item The wave functions $\psi_s$ change such that $\psi'_s(x') = \psi_s(x + \epsilon e^i)$ where $e^i$ is the unit vector in $i$ direction
    \item $\psi'_s(x') = \psi_s(x) + \epsilon {\partial \over \partial x_i} \psi_s(x) + O(\epsilon^2) = [I + \epsilon {\partial \over \partial x_i}] \psi_s \implies {i \over \hbar} \epsilon P_i = \epsilon {\partial \over \partial x_i} \implies P_i = -i \hbar {\partial \over \partial x_i}$
    \item $P_i = -i \hbar {\partial \over \partial x_i}$ is the generator of infinitesimal translations in $i$ direction
    % \item For an N-particle system, $P_i = -i \hbar \Sigma^{N -1}_{j = 0} {\partial \over \partial x_{3j + i}}$ because the $i^{th}$ coordinate ($x_{3j + i}$) of all particles change
    \item It can be shown that $\bra{\psi_1}\ket{P_i \psi_2} = \bra{P_i \psi_1} \ket{\psi_2} \implies P_i$ is Hermitian 
    \item $P_i$ is Hermitian $\implies U_i(\epsilon) = I + {i \over \hbar} \epsilon P_i$ is a symmetry transformation that preserves the system state
    \item Space translation is a symmetry operation only if the translation is along a cartesian coordinate dimension $\implies P_i$ is the generator of translation only in cartesian coordinates, for other coordinate systems $P_\eta \neq -i \hbar {\partial \over \partial \eta}$
    \item The finite translation transformation in the $i$ direction is $U_i(\alpha) = \exp(\alpha {\partial \over \partial x_i})$
    \item $[X_i,P_j]f = -i \hbar x_i {\partial f \over \partial x_j} + i \hbar {\partial \over \partial x_j} x_i f = i \hbar \delta_{ij} f \implies [X_i,P_j] = i \hbar \delta_{ij}$
  \end{itemize}
\end{frame}

\begin{frame}
  \frametitle{Space Rotation Generator}
  \vspace{1mm}
  A rotation by an infinitesimal angle $\alpha$ in space about the $i$ axis changes the space coordinates and the spin direction ($S^n \neq S^{n'}$ for any spin operator in the direction $\hat{n}$ because $\hat{n}' \neq \hat{n}$) $\implies \ket{\psi'} = U_{xi}(\epsilon) U_{si}(\epsilon) \ket{\psi}$
  \begin{itemize}
    \item The wave functions $\psi_s$ change such that $\psi'_s(x') = \psi_s(x + \alpha \epsilon_{ilm} x_l e^m e^i)$, $e^p$ is the unit vector in $p$ direction
    \item $\psi'_s(x') = \psi_s(x) + \alpha \epsilon_{ilm} x_l {\partial \over \partial x_m} \psi_s(x_j) + O(\alpha^2) = [I + \alpha \epsilon_{ilm} x_l {\partial \over \partial x_m}] \psi_s \implies {i \over \hbar} \alpha L_i = \alpha \epsilon_{ilm} x_l {\partial \over \partial x_m}$
    \item $L_i = -i \hbar \epsilon_{ilm} x_l {\partial \over \partial x_m}$ is the generator of infinitesimal rotations about $i$ axis for wave functions $\psi_s(x_i)$
    \item Since $P_i = -i \hbar {\partial \over \partial x_m}$, then $L_i = \epsilon_{ilm} X_l P_m$
    % \item N-particle system $L_i = \Sigma^{N -1}_{j = 0} \epsilon_{(3j + i)lm} X_l P_m$ because coordinates of all particles rotate about the $i^{th}$ axis
    \item It can be shown that $\bra{\psi_1}\ket{L_i \psi_2} = \bra{L_i \psi_1} \ket{\psi_2} \implies L_i$ is Hermitian 
    \item $L_i$ is Hermitian $\implies U_i(\alpha) = I + {i \over \hbar} \alpha L_i$ is a symmetry transformation that preserves the system state
    \item The finite rotation transformation about $i$ axis is $U_i(\beta) = \exp(\beta \epsilon_{ilm} x_l {\partial \over \partial x_m})$
    \item $[X_i,L_j] = [X_i,\epsilon_{jlm} X_l P_m] = \epsilon_{jlm} (X_i X_l P_m - X_l P_m X_i) = \epsilon_{jlm} X_l [X_i,P_m] = \epsilon_{jlm} X_l \delta_{im} i \hbar = i \hbar \epsilon_{ijl} X_l$
    \item The form of the spin transform $U_{si}$ is still to be determined $\implies \ket{\psi'} = [I + {i \over \hbar} \alpha L_i] U_{si}(\alpha) \ket{\psi}$
    \item The overall transformation is $U_{xi}(\alpha) U_{si}(\alpha) = [I + {i \over \hbar} \alpha L_i] [I + {i \over \hbar} \alpha S_i] = [I + {i \over \hbar} \alpha (L_i + S_i)] = [I + {i \over \hbar} \alpha J_i]$ where $J_i = L_i + S_i$
  \end{itemize}
\end{frame}

\begin{frame}
  \frametitle{Commutators of IT Generators}
  \vspace{1mm}
  Take two invarance transformations $\exp(i \alpha_1 K_1)$ and $\exp(i \alpha_2 K_2)$, and let $U_{12}$ be the product of the transformations and their inverses
  \begin{itemize}
    \item $U_{12} = \exp(i \alpha_1 K_1) \exp(i \alpha_2 K_2) \exp(-i \alpha_1 K_1) \exp(-i \alpha_2 K_2)$
    \item By expanding $U$ in powers of $\alpha_i$ and keeping only up to the second order terms, $U_{12} \approx I + \alpha_1 \alpha_2 [K_2,K_1]$
  \end{itemize}
  The infinitesimal transformation $U_{12} \approx I + \alpha_1 \alpha_2 [K_2,K_1]$ is equivalent to the application of $10$ invariance transformations (IT) up to an arbitrary phase factor $\exp(i \delta)$
  \begin{itemize}
    \item $U_{12} = \exp(i \delta) \Pi_{i = 1}^{10} \exp(i \beta_j K_j) \approx \exp(i \delta) (I + i \Sigma_{j = 1}^{10} \beta_j K_j)$ for infinitesimal $\alpha_1$ and $\alpha_2$
    \item $(1 - i \delta)(I + \alpha_1 \alpha_2 [K_2,K_1]) = I + i \Sigma_{j = 1}^{10} \beta_j K_j \implies [K_2,K_1] =  i \Sigma_{j = 1}^{10} {\beta_j \over \alpha_1 \alpha_2} K_j + i \delta I$
    \item Hence, the commutator of any two IT generators is a linear combination of all other IT generators
    \item Since this derivation does not depend on particular generators, it holds for the general case of any two IT generators
    \item The extra term $i \delta I$ is a multiple of identity that accounts for any phase factors
  \end{itemize}
  This is only true because the IT generators are independent of the transformation parameter $\alpha$, otherwise, the relationship between commutators and generators would be non-linear
\end{frame}

\begin{frame}
  \frametitle{Commutators of IT Generators II}
  \vspace{1mm}
  Any generator (operator, in general) commutes with itself $\implies [P_i,P_i] = [J_i,J_i] = [G_i,G_i] = [H,H] = 0$

  For two space rotations about axes $i$ and $j$: $\tau_1(R_i,0,0,0)$, $\tau_2(R_j,0,0,0)$
  \begin{itemize}
    \item $U_{12}(\tau^{-1}_1 \tau^{-1}_2 \tau_1 \tau_2) = \exp(i \theta_i J_i) \exp(i \theta_j J_j) \exp(-i \theta_i J_i) \exp(-i \theta_j J_j) \to I + \theta_i \theta_j [J_j,J_i]$
    \item But $\tau^{-1}_1 \tau^{-1}_2 \tau_1 \tau_2 = \tau(R^{-1}_i R^{-1}_j R_i R_j,0,0,0) \implies U_{12}(\tau^{-1}_1 \tau^{-1}_2 \tau_1 \tau_2) = U(R^{-1}_i R^{-1}_j R_i R_j,0,0,0)$
    \item For infinitesimal rotations about axis $i$, $R_i(\epsilon) = I + {i \over \hbar} \epsilon M^i$ and $R^{-1}_{i}(\epsilon) = I - {i \over \hbar} \epsilon M^i$
    \item Hence, $R_i(\epsilon_1) R_j(\epsilon_2) = I + {i \over \hbar} (\epsilon_1 M^{i} + \epsilon_2 M^{j}) - {1 \over \hbar^2}({\epsilon^2_1 \over 2} M^i M^i + \epsilon_1 \epsilon_2 M^i M^j + {\epsilon^2_2 \over 2} M^j M^j)$
    \item Similarly, $R^{-1}_{i}(\epsilon_1) R^{-1}_{j}(\epsilon_2) = I - {i \over \hbar} (\epsilon_1 M^{i} + \epsilon_2 M^{j}) - {1 \over \hbar^2}({\epsilon^2_1 \over 2} M^i M^i + \epsilon_1 \epsilon_2 M^i M^j + {\epsilon^2_2 \over 2} M^j M^j)$
    \item Hence, $R^{-1}_{i} R^{-1}_{j} R_i R_j = I + {\epsilon_1 \epsilon_2 \over \hbar^2} [M^j,M^i]$ up to a second order in $\epsilon$
    \item But $[M^i,M^j] = i \hbar \epsilon_{ijk} M^k \implies R^{-1}_{i} R^{-1}_{j} R_i R_j = I - i {\epsilon_1 \epsilon_2 \over \hbar} \epsilon_{ijk} M^k$
    \item Since $R_{k}(-\epsilon_1 \epsilon_2) = I - i {\epsilon_1 \epsilon_2 \over \hbar} M^k \implies U_{12}(\tau^{-1}_1 \tau^{-1}_2 \tau_1 \tau_2) = U(\tau(-\epsilon_{ijk} R_k,0,0,0))$
    \item Hence, $[J_j,J_i] = i \Sigma_{j = 1}^{10} \beta_j K_j + i \delta I \implies [J_j,J_i] = -i \epsilon_{ijk} J_k + i \delta I \implies [J_i,J_j] = i \epsilon_{ijk} J_k - i \delta I$
    \item $[J_i,J_j] = i \epsilon_{ijk} J_k - i \delta I = -[J_j,J_i] = -i \epsilon_{jik} J_k + i \delta I = i \epsilon_{ijk} J_k + i \delta I \implies \delta = 0 \implies [J_i,J_j] = i \epsilon_{ijk} J_k$
  \end{itemize}
\end{frame}

\begin{frame}
  \frametitle{Commutators of IT Generators III}
  \vspace{1mm}
  For two space translations: $\tau_1(I,d_i,0,0)$, $\tau_2(I,d_j,0,0)$ in any directions $i$, $j$
  \begin{itemize}
    \item $U_{12}(\tau^{-1}_1 \tau^{-1}_2 \tau_1 \tau_2) = \exp(i d_i P_i) \exp(i d_j P_j) \exp(-i d_i P_i) \exp(-i d_j P_j) \to I + d_i d_j [P_j,P_j]$
    \item But $\tau^{-1}_1 \tau^{-1}_2 \tau_1 \tau_2 = I \implies U_{12}(\tau^{-1}_1 \tau^{-1}_2 \tau_1 \tau_2) = I$
    \item Hence, $[P_j,P_i] = i \Sigma_{j = 1}^{10} \beta_j K_j + i \delta I \implies [P_j,P_i] = 0$ and $\delta$ can be arbitrarily chosen to be $0$
  \end{itemize}
  For a space translation in direction $i$ and a time translation: $\tau_1(I,d_i,0,0)$, $\tau_2(I,0,0,s)$
  \begin{itemize}
    \item $U_{12}(\tau^{-1}_1 \tau^{-1}_2 \tau_1 \tau_2) = \exp(i d_i P_i) \exp(-i s H) \exp(-i d_i P_i) \exp(i s H) \to I + s d_i [H,P_i]$
    \item But $\tau^{-1}_1 \tau^{-1}_2 \tau_1 \tau_2 = I \implies U_{12}(\tau^{-1}_1 \tau^{-1}_2 \tau_1 \tau_2) = I$
    \item Hence, $[H,P_i] = i \Sigma_{j = 1}^{10} \beta_j K_j + i \delta I \implies [H,P_i] = 0$ and $\delta$ can be arbitrarily chosen to be $0$
  \end{itemize}
  For two velocity transformations: $\tau_1(I,0,v_i,0)$, $\tau_2(I,0,v_j,0)$ in any directions $i$, $j$
  \begin{itemize}
    \item $U_{12}(\tau^{-1}_1 \tau^{-1}_2 \tau_1 \tau_2) = \exp(-i v_i G_i) \exp(-i v_j G_j) \exp(i v_i G_i) \exp(i v_j G_j) \to I + v_i v_j [G_j,G_i]$
    \item But $\tau^{-1}_1 \tau^{-1}_2 \tau_1 \tau_2 = I \implies U_{12}(\tau^{-1}_1 \tau^{-1}_2 \tau_1 \tau_2) = I$
    \item Hence, $[G_j,G_i] = i \Sigma_{j = 1}^{10} \beta_j K_j + i \delta I \implies [G_j,G_i] = 0$ and $\delta$ can be arbitrarily chosen to be $0$
  \end{itemize}
\end{frame}

\begin{frame}
  \frametitle{Commutators of IT Generators IV}
  \vspace{1mm}
  For a space rotation about axis $i$ and time translation: $\tau_1(R_i,0,0,0)$, $\tau_2(I,0,0,s)$
  \begin{itemize}
    \item $U_{12}(\tau^{-1}_1 \tau^{-1}_2 \tau_1 \tau_2) = \exp(i \theta_i J_i) \exp(-i s H) \exp(-i \theta_i J_i) \exp(i s H) \to I + \theta_i s [H,J_i]$
    \item But $\tau^{-1}_1 \tau^{-1}_2 \tau_1 \tau_2 = I \implies U_{12}(\tau^{-1}_1 \tau^{-1}_2 \tau_1 \tau_2) = I$
    \item Hence, $[H,J_i] = i \Sigma_{j = 1}^{10} \beta_j K_j + i \delta I \implies [H,J_i] = 0$ and $\delta$ can be arbitrarily chosen to be $0$
  \end{itemize}
  For any three operators $A$, $B$ and $C$: $[[A,B],C] = [[C,B],A] + [[A,C],B]$
  
  For a space rotation about axis $i$ and a space translation in direction $j$: $\tau_1(R_i,0,0,0)$, $\tau_2(I,d_j,0,0)$
  \begin{itemize}
    \item $U_{12}(\tau^{-1}_1 \tau^{-1}_2 \tau_1 \tau_2) = \exp(i \theta_i J_i) \exp(i d_j P_j) \exp(-i \theta_i J_i) \exp(-i d_j P_j) \to I + \theta_i d_j [P_j,J_i]$
    \item But $\tau^{-1}_1 \tau^{-1}_2 \tau_1 \tau_2 = \tau(I,(I - R^{-1}_i)d_j,0,0) \implies U_{12}(\tau^{-1}_1 \tau^{-1}_2 \tau_1 \tau_2) = U(I,(I - R^{-1}_i)d_j,0,0)$
    \item $U_{12}$ corresponds to a space translation in a direction normal to $i$ and $j$ 
    \item Hence, $[P_j,J_i] = i \Sigma_{j = 1}^{10} \beta_j K_j + i \delta I \implies [P_j,J_i] = i \epsilon_{jik} P_k + i \delta I \implies [J_i,P_j] = i \epsilon_{ijk} P_k + i \delta I$
    \item $[[J_1,J_2],P_3] = [[P_3,J_2],J_1] + [[J_1,P_3],J_2] \implies i[J_3,P_3] = -i[P_1,J_1] + i \delta [I,J_1] - i[P_2,J_2] + i \delta [I,J_1]$
    \item Hence, $[J_3,P_3] = [J_1,P_1] + [J_2,P_2]$, $[J_2,P_2] = [J_1,P_1] + [J_3,P_3]$, $[J_1,P_1] = [J_3,P_3] + [J_2,P_2]$
    \item This is only true if $[J_1,P_1] = [J_2,P_2] = [J_3,P_3] = 0 \implies \delta = 0 \implies [J_i,P_j] = i \epsilon_{ijk} P_k$
  \end{itemize}

\end{frame}

\begin{frame}
  \frametitle{Commutators of IT Generators V}
  \vspace{1mm}
    For a space rotation about axis $i$ and a velocity transformation in direction $j$: $\tau_1(R_i,0,0,0)$, $\tau_2(I,0,v_j,0)$
  \begin{itemize}
    \item $U_{12}(\tau^{-1}_1 \tau^{-1}_2 \tau_1 \tau_2) = \exp(i \theta_i J_i) \exp(-i v_j G_j) \exp(-i \theta_i J_i) \exp(i v_j G_j) \to I + \theta_i v_j [G_j,J_i]$
    \item But $\tau^{-1}_1 \tau^{-1}_2 \tau_1 \tau_2 = \tau(I,0,(I - R^{-1}_i)v_j,0) \implies U_{12}(\tau^{-1}_1 \tau^{-1}_2 \tau_1 \tau_2) = U(I,0,(I - R^{-1}_i)v_j,0)$
    \item $U_{12}$ corresponds to a velocity transformation in a direction normal to $i$ and $j$ 
    \item Hence, $[G_j,J_i] = i \Sigma_{j = 1}^{10} \beta_j K_j + i \delta I \implies [G_j,J_i] = i \epsilon_{jik} G_k + i \delta I  \implies [J_i,P_j] = i \epsilon_{ijk} P_k + i \delta I$
    \item $[[J_1,J_2],G_3] = [[G_3,J_2],J_1] + [[J_1,G_3],J_2] \implies i[J_3,G_3] = -i[G_1,J_1] + i \delta [I,J_1] - i[G_2,J_2] + i \delta [I,J_1]$
    \item Hence, $[J_3,G_3] = [J_1,G_1] + [J_2,G_2]$, $[J_2,G_2] = [J_1,G_1] + [J_3,G_3]$, $[J_1,G_1] = [J_3,G_3] + [J_2,G_2]$
    \item This is only true if $[J_1,G_1] = [J_2,G_2] = [J_3,G_3] = 0 \implies \delta = 0 \implies [J_i,G_j] = i \epsilon_{ijk} G_k$
  \end{itemize}
  For a velocity transformation about axis $i$ and time translation: $\tau_1(I,0,v_i,0)$, $\tau_2(I,0,0,s)$
  \begin{itemize}
    \item $U_{12}(\tau^{-1}_1 \tau^{-1}_2 \tau_1 \tau_2) = \exp(-i v_i G_i) \exp(-i s H) \exp(i v_i G_i) \exp(i s H) \to I + v_i s [H,G_i]$
    \item But $\tau^{-1}_1 \tau^{-1}_2 \tau_1 \tau_2 = \tau(I,v_i s,0,0) \implies U_{12}(\tau^{-1}_1 \tau^{-1}_2 \tau_1 \tau_2) = U(I,v_i s,0,0)$
    \item Hence, $[H,G_i] = i \Sigma_{j = 1}^{10} \beta_j K_j + i \delta I \implies [G_i,H] = i P_i + i \delta I$
    \item $[[J_1,G_2],H] = [[H,G_2],J_1] + [[J_1,H],G_2] \implies i[G_3,H] + i \delta [I,H] = -i[P_2,J_1] + i \delta [I,J_1] + [0,G_2]$
    \item Hence, $[G_3,H] = [J_1,P_2] = i P_3$, $[G_2,H] = i P_2$ and $[G_1,H] = i P_1 \implies \delta = 0 \implies [G_i,H] = i P_i$
  \end{itemize}
\end{frame}

\begin{frame}
  \frametitle{Commutators of IT Generators VI}
  \vspace{1mm}
  For a space translation in direction $i$ and velocity transformation in direction $j$: $\tau_1(I,d_i,0,0)$, $\tau_2(I,0,v_j,0)$
  \begin{itemize}
    \item $U_{12}(\tau^{-1}_1 \tau^{-1}_2 \tau_1 \tau_2) = \exp(i d_i P_i) \exp(-i v_j G_j) \exp(-i d_i P_i) \exp(i v_j G_j) \to I + d_i v_j [G_j,P_i]$
    \item But $\tau^{-1}_1 \tau^{-1}_2 \tau_1 \tau_2 = I \implies U_{12}(\tau^{-1}_1 \tau^{-1}_2 \tau_1 \tau_2) = I$
    \item Hence, $[G_j,P_i] = i \Sigma_{j = 1}^{10} \beta_j K_j + i \delta I \implies [G_j,P_i] = i \delta I$
    \item $[[J_1,G_2],P_1] = [[P_1,G_2],J_1] + [[J_1,P_1],G_2] \implies i[G_3,P_1] = i \delta [I,J_1] + [0,G_2] \implies [G_3,P_1] = 0$
    \item Similarly, $[G_3,P_2] = [G_2,P_1] = [G_2,P_3] = [G_1,P_2] = [G_1,P_3] = 0$
    \item Also $[[J_1,G_2],P_3] = [[P_3,G_2],J_1] + [[J_1,P_3],G_2] \implies i[G_3,P_3] = [0,J_1] -i [P_2,G_2]$
    \item Hence, $[G_3,P_3] = [G_2,P_2]$ and $[G_3,P_3] = [G_1,P_1] \implies [G_i,P_j] = i \delta_{ij} M I$
    \item $\delta = M$ only if $i = j$ in $[G_i,P_j]$, otherwise, $\delta = 0$
  \end{itemize}
  Here, $M$ cannot be arbitrarily chosen to be zero, unlike $[P_i,P_j]$, $[G_i,G_j]$, $[P_i,H]$ and $[J_i,H]$ where
  \begin{itemize}
    \item $[[J_2,P_3],H] = [[H,P_3],J_2] + [[J_2,H],P_3] \implies i[P_1,H] = 0 \implies [P_i,H] = 0$
    \item $[[J_2,P_3],P_2] = [[P_2,P_3],J_2] + [[J_2,P_2],P_3] \implies i[P_1,P_2] = 0 \implies [P_i,P_j] = 0$
    \item $[[J_2,G_3],G_2] = [[G_2,G_3],J_2] + [[J_2,G_2],G_3] \implies i[G_1,G_2] = 0 \implies [G_i,G_j] = 0$
    \item $[[J_2,J_3],H] = [[H,J_3],J_2] + [[J_2,H],J_3] \implies i[J_1,H] = 0 \implies [J_i,H] = 0$
  \end{itemize}
\end{frame}

\begin{frame}
  \frametitle{Commutators of IT Generators VII}
  \vspace{1mm}
  Since the space translation generator $P_i$ was found to be $P_i = -i \hbar {\partial \over \partial x_i}$
  \begin{itemize}
    \item $[P_i,P_j] = - \hbar^2 ({\partial \over \partial} x_i {\partial \over \partial x_j} - {\partial \over \partial x_j} {\partial \over \partial x_i}) = 0 \implies [P_i,P_j] = 0$ as derived from the IT commutator relations
  \end{itemize}
  For space rotation generators, $[J_i,J_j] = i \epsilon_{ijk} J_k$, $J_i = L_i + S_i$ and
  \begin{itemize}
    \item $[L_i,S_j] = 0$, $L_i = \epsilon_{ilm} X_l P_m$ and $[X_i,L_j] = i \hbar \epsilon_{ijl} X_l$
    \item $[P_i,L_j] = [P_i,\epsilon_{jlm} X_l P_m] = \epsilon_{jlm} (P_i X_l P_m - X_l P_m P_i) = \epsilon_{jlm} [P_i,X_l] P_m = -\epsilon_{jlm} P_m \delta_{il} i \hbar = i \hbar \epsilon_{ijl} P_l$
    \item $[L_i,L_j] = \epsilon_{ilm} [X_l P_m,L_j] = \epsilon_{ilm} (X_l [P_m,L_j] + [X_l,L_j] P_m) = i \hbar \epsilon_{ilm} (\epsilon_{mjq} X_l P_q + \epsilon_{ljq} X_q P_m)$
    \item $[L_i,L_j] = i \hbar (X_i P_j - X_j P_i)$ and $\epsilon_{ijk} L_k = \epsilon_{ijk} \epsilon_{klm} X_l P_m = X_i P_j - X_j P_i$, then $[L_i,L_j] = i \epsilon_{ijk} L_k$
    \item $[J_i,J_j] = [L_i + S_i,L_j + S_j] = [L_i,L_j] + [S_i,L_j] + [L_i,S_j] + [S_i,S_j] = i \epsilon_{ijk} L_k + [S_i,S_j] = i \epsilon_{ijk} (L_k + S_k)$
    \item Then, $[S_i,S_j] = i \epsilon_{ijk} S_k$
  \end{itemize}
  This result is true whether $S_i$ is a discrete or a continuous generator
  \begin{itemize}
    \item $[X_i,L_j] = i \hbar \epsilon_{ijl} X_l \implies [X_i,J_j] = [X_i,L_j + S_j] = i \hbar \epsilon_{ijl} X_l + [X_i,S_j]$
    \item $S_i$ are generators that handle the transformation of spin wave functions $\psi_s(x)$, they are related to the spin operator $S^n$ in the $\hat{n}$ direction as it will be derived
  \end{itemize}
\end{frame}

\begin{frame}
  \frametitle{Position Operator Commutators}
  \vspace{1mm}
  The position operator $X_i$ multiplies any ket $\ket{\psi}$ by its space coordinate $x_i \implies X_i \ket{\psi} = x_i \ket{\psi}$. 

  For any unitary transformation generator $K_j$
  \begin{itemize}
    \item The corresponding unitary transformation is $U(\alpha) = \exp({i \over \hbar} \alpha K_j)$ for a transformation parameter $\alpha$
    \item The transformed position operator $X'_i = U X_i U^{\dagger} = \exp({i \over \hbar} \alpha K_j) X_i \exp(-{i \over \hbar} \alpha K_j) = X_i + {i \over \hbar} \alpha [X_i,K_j] + O(\alpha^2)$
    \item The transformed position eigenkets $\ket{x'} = U \ket{x} = \exp({i \over \hbar} \alpha K_j) \ket{x}$
  \end{itemize}
  For a space rotation about $j$ axis, the generator is $J_j \implies X'_i = X_i + {i \over \hbar} \alpha [X_i,J_j] + O(\alpha^2)$
  \begin{itemize}
    \item A point $x$ transforms to $x + \alpha r \times e^j \implies x_k \to x_k + \alpha \epsilon_{lmk} x_l e^j_m$ where $e^j$ is the unit vector in $j$ direction upon a rotation by an infinitesimal angle $\alpha$
    \item $X'_i$ measures the position relative to the rotated coordinates $\implies X_i \ket{x} = x_i \ket{x}$ and $X'_i \ket{x'} = x_i \ket{x'}$
    \item $\ket{x'} = \exp({i \over \hbar} \alpha J_j) \ket{x} = \ket{x_k - \alpha \epsilon_{lmk} x_l \delta_{jm}}$ since $e^j_m = \delta_{jm}$ (opposite direction transformation)
    \item $X'_i \ket{x'} = (X_i + {i \over \hbar} \alpha [X_i,J_j])\ket{x'} = (x_i - \alpha \epsilon_{lmi} x_l \delta_{mj})\ket{x'} + {i \over \hbar} \alpha [X_i,J_j] \ket{x'} \implies -\alpha \epsilon_{lji} x_l + {i \over \hbar} \alpha [X_i,J_j] = 0$
    \item Hence, $[X_i,J_j] = -i \hbar \epsilon_{lji} x_l I = i \hbar \epsilon_{ijl} X_l \implies [J_i,X_j] = i \hbar \epsilon_{ijl} X_l$
  \end{itemize}
  But $[X_i,J_j] = i \hbar \epsilon_{ijl} X_l + [X_i,S_j] = i \hbar \epsilon_{ijl} X_l \implies [X_i,S_j] = 0$ 
\end{frame}

% \begin{frame}
%   \frametitle{Position Operator Commutators II}
%   \vspace{1mm}
%   For a velocity transformation in $j$ direction, the generator is $G^t_j \implies X'_i = X_i + i \alpha [X_i,G^t_j] + O(\alpha^2)$
%   \begin{itemize}
%     \item $G^t_j$ is parametrized by $t$ because it transforms space coordinates to ones of a moving coordinate system at time $t$. It is assumed here that origins coincide at $t = 0$
%     \item $X'_i$ measures the position relative to the moving coordinates $\implies X_i \ket{x} = x_i \ket{x}$ and $X'_i \ket{x'} = x_i \ket{x'}$
%     \item $\ket{x'} = \exp(i \alpha G^t_j) \ket{x} = \ket{x + \alpha t e^j}$ where $e^j_p = \delta_{jp}$ is unit vector in $j$ direction
%     \item $X'_i \ket{x'} = (X_i + i \alpha [X_i,G^t_j])\ket{x'} = (x_i + \alpha t \delta_{ij})\ket{x'} + i \alpha [X_i,G^t_j] \ket{x'} \implies \alpha t \delta_{ij} + i \alpha [X_i,G^t_j] = 0$
%     \item Hence, $[X_i,G^t_j] = i \delta_{ij} t I$
%     \item $$
%   \end{itemize}
% \end{frame}

\begin{frame}
  \frametitle{Rotation Generator and Spin***}
  \vspace{1mm}
  The rotation generator $J_k = L_k + S_k = \epsilon_{ijk} X_i P_j + S_k$
  \begin{itemize}
    \item It was shown that $[J_i,X_j] = i \epsilon_{ijk} X_k,[J_i,P_j] = i \epsilon_{ijk} P_k$, $[X_i,P_j] = i \delta_{ij} I$ and $[S_i,S_j] = i \epsilon_{ijk} S_k$
    \item $[S_i,X_j] = [S_i,L_j] = 0$ \implies [S_i,L_j] = \epsilon_{jlm} [S_i,X_l P_m] = \epsilon_{jlm} (X_l [S_i,P_m] + [S_i,X_l] P_m) = \epsilon_{jlm} X_l [S_i,P_m]$
    \item Then $J_k = \epsilon_{ijk} X_i P_j + S_k + c_k I$ where $c_k$ is a scalar satisfies all $J_i$ commutators conditions
    \item $[J_i,X_j] = \epsilon_{lmi} [X_l P_m,X_j] + c_i [I,X_j] + [S_i,X_j] = \epsilon_{lmi} X_l [P_m,X_j] = -\epsilon_{lmi} X_l \delta_{mj} = \epsilon_{ijk} X_k$ 
    \item $[J_i,P_j] = \epsilon_{lmi} [X_l P_m,P_j] + c_i [I,X_j] + [S_i,X_j] = \epsilon_{lmi} [X_l,P_j] P_m = \epsilon_{lmi} P_m \delta_{lj} = \epsilon_{ijk} P_k$
    \item $[J_i,J_j] = \epsilon_{pqi} (\epsilon_{lmj} [X_p P_q,X_l P_m] + [X_p P_q,S_j] + c_j [X_p P_q,I]) + \epsilon_{lmj} [S_i,X_l P_m] + [S_i,S_j] + c_j [S_i,I] + c_i (\epsilon_{lmj} [I,X_l P_m] + [I,S_j] + [I,I])$
    \item $[J_i,J_j] = \epsilon_{pqi} \epsilon_{lmj} \{X_p([P_q,X_l] P_m + X_l [P_q,P_m]) + (X_l [X_p ,P_m] + [X_p ,X_l] P_m)P_q\} + i \epsilon_{ijk} S_k$
    \item $[J_i,J_j] = i \epsilon_{pqi} \epsilon_{lmj} \{-\delta_{ql} X_p P_m + \delta_{pm} X_l P_q\} + i \epsilon_{ijk} S_k = i \epsilon_{ijk} X_i P_j + i \epsilon_{ijk} S_k = J_k$ if and only if $c_i = 0$
    \item Hence, $J_k = \epsilon_{ijk} X_i P_j + S_k$
  \end{itemize}
  Since $S_i$ commutes with $X_i$, then $S_i$ can be used to represent any degrees of freedom (DoF) that are independent of $X_i$. All such DoFs will be called spin DoFs
  \begin{itemize}
    \item The spin DoFs form a separate Hilbert space, the tensor product of which with the position DoFs is another Hilbert space with basis kets $\ket{x_1 x_2 ... x_n s_1 s_2 ... s_m}$
    \item Operators acting on position DoFs commute with those acting on spin DoFs
    \item No physical significance has been given to spin DoFs so far
  \end{itemize}
\end{frame}

\begin{frame}
  \frametitle{Ehrenfest Theorem}
  \vspace{1mm}
  The time rate of change of the expectation of any operator $A$ is
  \begin{itemize}
    \item $\dot{\expval{A}} = {d \over dt} \expval{A}{\psi(t)} = ({d \over dt} \bra{\psi}) A \ket{\psi} + \bra{\psi} A ({d \over dt} \ket{\psi})$
    \item $\dot{\expval{A}} = ({i \over \hbar} \bra{\psi} H^{\dagger}) A \ket{\psi} + \bra{\psi} A ({-i \over \hbar} H \ket{\psi}) = {i \over \hbar} \expval {H^{\dagger} A - A H}{\psi} = {i \over \hbar} \expval {[H,A]}{\psi}$
  \end{itemize}
  Hence, if the dynamical variables $b$ and $a$ are related such that $b = {da \over dt} \implies B = {i \over \hbar} [H,A]$
  \begin{itemize}
    \item For the case of the position operator $X_i$, the velocity operator $V_i = {i \over \hbar} [H,X_i]$
  \end{itemize}
\end{frame}

\begin{frame}
  \frametitle{Commutators of Triples of Operators}
  \vspace{1mm}
  Take any set of three operators that are logically related in some manner, for example (for $i \in \{1,2,3\}$)
  \begin{itemize}
    \item The set of position operators $X_i$
    \item The set of momentum operators $P_i$
    \item The set of orbital angular momentum operators $L_i$
    \item The set of spin angular momentum operators $S_i$
  \end{itemize}
  This case is important for dynamical variables that have components in all three dimensions

  The operators can either
  \begin{itemize}
    \item I: Commute, such that their commutators vanish $[K_i,K_j] = 0$
    \item II: Not commute, such that their commutators are multiples of identity $[K_i,K_j] = i c_{ij} I$
    \item III: Not commute, such that their commutators are other commutators from the same set $[K_i,K_j] = i c \epsilon_{ijk} K_k$
    \item IV: Follow other commutation relations
  \end{itemize}
  For case III, the multiplier does not depend on the commutator's operator pair since space is assumed to be isotropic. 
\end{frame}

\begin{frame}
  \frametitle{Commutators of Case III Operators}
  \vspace{1mm}
  For the case of $[K_i,K_j] = c \epsilon_{ijk} K_k$, define $K^2 = K_i K_i$, $K_{+} = K_1 + i K_2$ and $K_{-} = K_1 - i K_2$
  \begin{itemize}
    \item $[K^2,K_i] = [K_j K_j,K_i] = K_j [K_j,K_i] + [K_j,K_i]K_j = c(\epsilon_{jik} K_j K_k + \epsilon_{kij} K_j K_k) = 0$
    \item $[K_3,K_{+}] = [K_3,K_1] + i[K_3,K_2] = c(iK_2 - i^2 K_1) = c K_{+}$
    \item $[K_3,K_{-}] = [K_3,K_1] - i[K_3,K_2] = c(iK_2 - i^2 K_1) = -c K_{-}$
    \item $[K_{+},K_{-}] = [K_1,K_1] - i[K_1,K_2] + i[K_2,K_1] + [K_2,K_2] = c(K_3 + K_3) = 2cK_3$
    \item $[K^2,K_{+}] = [K^2,K_{-}] = 0$
    \item $K_{-} K_{+} = K_{1}^{2} - i K_2 K_1  + i K_2 K_1 + K_{2}^{2} = K_{1}^{2} + K_{2}^{2} + i[K_1,K_2] = K_{1}^{2} + K_{2}^{2} + i^2 c K_3 = K^{2} - K_{3}^{2} - c K_3$
    \item $K_{+} K_{-} = K_{1}^{2} + i K_2 K_1  - i K_2 K_1 + K_{2}^{2} = K_{1}^{2} + K_{2}^{2} + i[K_2,K_1] = K_{1}^{2} + K_{2}^{2} - i^2 c K_3 = K^{2} - K_{3}^{2} + c K_3$
  \end{itemize}
  Furthermore, if $K_i$ are Hermitian, then $K_i^{\dagger} = K_i \implies K_{+}^{\dagger} = K_{-}$ and $K_{-}^{\dagger} = K_{+}$
  
  Since $[K^2,K_i] = 0$ for all $i$ but $[K_i,K_j] \neq 0$ for all $i,j$ pairs
  \begin{itemize}
    \item There exists a set of common eigenkets for $K^2$ and each one of the $K_i$ operators
    \item However, there are no common eigenkets between $K_i$ and $K_j$ for $i \neq j$
    \item This may indicate that $K^2$ is degenerate and has multiple eigenkets for the same eigenvalue with the different eigenkets corresponding to the same eigenvalue being also eigenkets for $K_1$, $K_2$ or $K_3$
  \end{itemize}
\end{frame}

\begin{frame}
  \frametitle{Case III Operators Eigenkets}
  \vspace{1mm}
  For the case of $[K_i,K_j] = c \epsilon_{ijk} K_k$, there exists common eigenkets for $K^2$ and each $K_i$

  Let $\ket{\beta,\alpha}$ be an eigenket for $K^2$ and $K_3$ with eigenvalues $\beta$ and $\alpha$ respectively
  \begin{itemize}
    \item $K^2 \ket{\beta,\alpha} = \beta \ket{\beta,\alpha}$ and $K_3 \ket{\beta,\alpha} = \alpha \ket{\beta,\alpha}$
    \item $K_3 K_{+} \ket{\beta,\alpha} = (K_{+} K_3 + c K_{+}) \ket{\beta,\alpha} = (\alpha + c) K_{+} \ket{\beta,\alpha}$
    \item $K_3 K_{-} \ket{\beta,\alpha} = (K_{-} K_3 - c K_{-}) \ket{\beta,\alpha} = (\alpha - c) K_{-} \ket{\beta,\alpha}$
    \item $K^2 K_{+} \ket{\beta,\alpha} = K_{+} K^2 \ket{\beta,\alpha} = \beta K_{+} \ket{\beta,\alpha}$
    \item $K^2 K_{-} \ket{\beta,\alpha} = K_{-} K^2 \ket{\beta,\alpha} = \beta K_{+} \ket{\beta,\alpha}$
    \item $K_{+} \ket{\beta,\alpha}$ and $K_{-} \ket{\beta,\alpha}$ are eigenkets for $K_3$ with eigenvalues $\alpha + c$ and $\alpha - c$ respectively
    \item $K_{+} \ket{\beta,\alpha}$ and $K_{-} \ket{\beta,\alpha}$ are eigenkets for $K^2$ with eigenvalues $\beta \implies K^2$ is a degenerate operator
    \item Hence, $K_{+}$ is a raising operator that transforms $K_3$ eigenkets to ones with an  eigenvalue that is $c$ higher
    \item Similarly, $K_{-}$ is a lowering operator that transforms $K_3$ eigenkets to ones with an  eigenvalue that is $c$ lower
  \end{itemize}
  Same results hold for eigenkets common to $\{K^2$,K_1\}$ and $\{K^2,K_2\}$
\end{frame}

\begin{frame}
  \frametitle{Case III Operators Eigenkets II}
  \vspace{1mm}
  For Hermitian $K_i$, $K_{2}$ is also Hermitian $\implies \alpha,\beta$ are real
  \begin{itemize}
    \item $\expval{K^2}{\beta,\alpha} = \expval{K_{1}^{2}}{\beta,\alpha} + \expval{K_{2}^{2}}{\beta,\alpha} + \expval{K_{3}^{2}}{\beta,\alpha}$
    \item But $\expval{K_{1}^{2}}{\beta,\alpha} = \expval{K_1 K_1}{\beta,\alpha} = \expval{K_{1}^{\dagger} K_1}{\beta,\alpha} = (\bra{\beta,\alpha} K_{1}^{\dagger}) (K_1 \ket{\beta,\alpha}) > 0$
    \item Similarly, $\expval{K_{2}^{2}}{\beta,\alpha} > 0$
    \item Hence, $\expval{K^2}{\beta,\alpha} > \expval{K_{3}^{2}}{\beta,\alpha} \implies \beta \braket{{\beta,\alpha}} > \alpha^2 \braket{{\beta,\alpha}} \implies \beta > \alpha^2$
    \item For any $K^2$ eigenvalue $\beta$, there may exist multiple $K_3$ eigenkets with eigenvalues $\alpha \in [-\sqrt{\beta},\sqrt{\beta}]$ 
    \item There exist $\alpha_{+} > \sqrt{\beta} - c$ and $\alpha_{-} < \sqrt{\beta} + c$ such that $K_{+} \ket{\beta,\alpha_{+}} = K_{-} \ket{\beta,\alpha_{-}} = 0$
    \item Otherwise, the raising operator $K_{+}$ would transform $\ket{\beta,\alpha_{+}}$ to $K_3$ eigenket with $\alpha > \sqrt{\beta} \implies \alpha > \beta^2$
    \item Similarly, the lowering operator $K_{-}$ would transform $\ket{\beta,\alpha_{-}}$ to $K_3$ eigenket with $\alpha < -\sqrt{\beta} \implies \alpha^2 > \beta$
    \item For $K_{+} \ket{\beta,\alpha_{+}} = 0$, $K_{-} K_{+} \ket{\beta,\alpha_{+}} = (K^{2} - K_{3}^{2} - c K_3) \ket{\beta,\alpha_{+}} = (\beta - \alpha_{+}^2 - c \alpha_{+}) \ket{\beta,\alpha_{+}} = 0$
    \item Since $\ket{\beta,\alpha_{+}} \neq 0 \implies \beta = \alpha_{+}(\alpha_{+} + c)$, similarly, $\beta = \alpha_{-}(\alpha_{-} - c)$
    \item By solving the equations $\alpha_{+} = {-c + \sqrt{c^2 + 4 \beta} \over 2}$ and $\alpha_{-} = {c - \sqrt{c^2 + 4 \beta} \over 2} \implies \alpha_{+} = -\alpha_{-}$
  \end{itemize}
  This analysis shows that like $\alpha$ and $\beta$, $c$ is also real, or else, some of $K_3$ eigenvalues would be complex
\end{frame}

\begin{frame}
  \frametitle{Case III Operators Eigenkets III}
  \vspace{1mm}
  For a given $K^2$ eigenvalue $\beta$, starting from the $K_3$ eigenvalue $\alpha_{+} = {-c + \sqrt{c^2 + 4 \beta} \over 2}$
  \begin{itemize}
    \item By applying $L_{-}$ on $\ket{\beta,\alpha_{+}}$ successively, eigenkets with lower eigenvalues are obtained in steps of $c$
    \item After $n$ applications of $L_{-}$ on $\ket{\beta,\alpha_{+}}$, the eigenket $\ket{\beta,\alpha_{-}}$ with $K_3$ eigenvalue $\alpha_{-}$ is obtained
    \item Hence, $\alpha_{+} - \alpha_{-}$ is an integer multiple of $c \implies \alpha_{+} - \alpha_{-} = nc \implies {-c + \sqrt{c^2 + 4 \beta} \over 2} - {c - \sqrt{c^2 + 4 \beta} \over 2} = nc$
    \item $c^2 + 4 \beta = (n + 1)^2 c^2 \implies \beta = c^2 {n(n + 2) \over 4}$ and $\alpha_{+} = -\alpha_{-} = {nc \over 2}$
  \end{itemize}
  These results mean that for a given non-negative integer $n$
  \begin{itemize}
    \item There is a $K^2$ eigenvalue $\beta = c^2 {n(n + 2) \over 4} \implies K^2$ eigenvalues are quantized
    \item $K_3$ eigenvalues are quantized with eigenvalues in steps of $c \over 2$ in the range $[-{nc\over 2},{nc \over 2}]$
    \item There are exactly $n + 1$ eigenvalues for $K_3$ with eigenkets common with $K^2$
    \item The degeneracy of $K^2$ eigenvalue $\beta$ is $3 \times (n + 1)$ since there are $n + 1$ common eigenkets with $K_1$, $K_2$ and $K_3$
  \end{itemize}
  Note that $K^2$ and $K_i$ quantization follows from the commutation relations between $K_i$ and their Hermiticity only without resorting to any particular system, boundary condittions or even to the nature of $K_i$
\end{frame}

\begin{frame}
  \frametitle{Case III Operators Matrices}
  \vspace{1mm}
  Switch the $K^2$ eigenket labeling from $\ket{\beta,\alpha}$ to $\ket{k,m}$
  \begin{itemize}
    \item Here, $k = {n \over 2}$, $\beta = c^2 k(k + 1)$ and $\alpha = mc$ where $-k \le m \le k$
    \item It follows from the new labeling that $K_{+} \ket{k,m} = A \ket{k,m+1}$ and $K_{-} \ket{k,m} = B \ket{k,m-1}$
    \item $A$ and $B$ are normalization constants such that $A^{*}A \braket{k,m+1}{k,m+1} = B^{*}B \braket{k,m-1}{k,m-1} = 1$
    \item Let $T = A^{*}A \braket{k,m+1}{k,m+1} = \bra{k,m} K^{\dagger}_{+} K_{+} \ket{k,m} = \bra{k,m} K_{-} K_{+} \ket{k,m} = A^{*}A = |A|^2$
    \item $T = \bra{k,m} (K^{2} - K_{3}^{2} - c K_3) \ket{k,m} = \bra{k,m}(c^2{k(k + 1)} - c^2 m^2 - c^2m)\ket{k,m} = c^2[k(k + 1) - m(m + 1)]$
    \item $|A|^2 = c^2[k(k + 1) - m(m + 1)] \implies A = c[k(k + 1) - m(m + 1)]^{1/2}$
    \item Similarly, $B = c[k(k + 1) - m(m - 1)]^{1/2}$
    \item $K_{+} \ket{k,m} = c[k(k + 1) - m(m + 1)]^{1/2} \ket{k,m + 1}$, $K_{-} \ket{k,m} = c[k(k + 1) - m(m - 1)]^{1/2} \ket{k,m - 1}$
    \item The pases of $A$ and $B$ can be arbitrarily chosen and do not affect the state $\implies$ choose them to be $0$
  \end{itemize}
  Taking the $\ket{k,m}$ eigenkets common to $K^2$ and $K_3$ as a basis, the matrix elements of $K_3$ operator are
  \begin{itemize}
    \item $\bra{k',m'} K^2 K_3 \ket{k,m} = k'(k' + 1)c^2 \bra{k',m'} K_3 \ket{k,m}$
    \item $\bra{k',m'} K_3 K^2 \ket{k,m} = \bra{k',m'} K_3 \ket{k,m} k(k + 1)c^2$
    \item Since $K_3$: $\bra{k',m'} K^2 K_3 \ket{k,m} = \bra{k',m'} K_3 K^2 \ket{k,m} \implies \bra{k',m'} K_3 \ket{k,m}$ if $j \neq j'$
  \end{itemize}
\end{frame}

\begin{frame}
  \frametitle{Case III Operators Matrices II}
  \vspace{1mm}
  In the $\ket{k,m}$ eigenkets basis
  \begin{itemize}
    \item $\bra{k',m'} K_3 \ket{k,m}$ if $k \neq k' \implies \bra{k',m'} K_3 \ket{k,m} = mc \delta_{k,k'}\delta_{m,m'}$
    \item Same result holds for matrices of $K_1$ and $K_2$ operators in their eigenket bases
  \end{itemize}
  For operators $K_{+}$, $K_{-}$, $K_1 = {K_{+} + K_{-} \over 2}$ and $K_2 = {K_{+} - K_{-} \over 2i}$
  \begin{itemize}
    \item $\bra{k',m'} K_{+} \ket{k,m} = \bra{k',m'} c[k(k + 1) - m(m + 1)]^{1/2} \ket{k,m - 1} = c[k(k + 1) - m(m + 1)]^{1/2} \delta_{k,k'} \delta_{m',m + 1}$
    \item $\bra{k',m'} K_{-} \ket{k,m} = \bra{k',m'} c[k(k + 1) - m(m - 1)]^{1/2} \ket{k,m - 1} = c[k(k + 1) - m(m - 1)]^{1/2} \delta_{k,k'} \delta_{m',m - 1}$
    \item Operator $K_1$: ${c \delta_{k,k'} \over 2} ([k(k + 1) - m(m + 1)]^{1/2} \delta_{m',m + 1} + [k(k + 1) - m(m - 1)]^{1/2} \delta_{m',m - 1})$
    \item Operator $K_2$: ${c \delta_{k,k'} \over 2i} ([k(k + 1) - m(m + 1)]^{1/2} \delta_{m',m + 1} - [k(k + 1) - m(m - 1)]^{1/2} \delta_{m',m - 1})$
  \end{itemize}
  This shows that all operator matrices are diagonal in $k$
  \begin{itemize}
    \item $K^2$ and $K_3$ operator matrices are also diagonal $m$
    \item $K_1$ and $K_2$ operator matrices are $1$ off-diagonal in $m$
  \end{itemize}
  These results are very general and hold for all Hermitian operators $K_i$ that satisfy the commutation relation $[K_i,K_j] = i c \epsilon_{ijk} K_k$
\end{frame}

\begin{frame}
  \frametitle{Orbital Angular Momentum Operators}
  \vspace{1mm}
  Applying an infinitesimal rotation to space using the transformation $T_k(\epsilon)$ results in another transformation $U_k(\epsilon)$ being applied to the Hilbert space kets $\ket{\psi}$
  \begin{itemize}
    \item $U_k(\epsilon) \to I$ as $\epsilon \to 0 \implies U_k(\epsilon)$ is linear in $\epsilon$
    \item $U_k(\epsilon)$ has to be unitary to preserve normalization of space kets $\ket{\psi}$
    \item Postulate a transformation $U_k(\epsilon) = I + {i \over \hbar} \epsilon L_k$ where $J_k$ is a Hermitian operator 
    \item $U^{\dagger}_k(\epsilon) = I - {i \over \hbar} \epsilon J_k \implies U_k(\epsilon) U^{\dagger}_k(\epsilon) = I - {\epsilon^{2} \over \hbar^2} L_k L_k \to I$ as $\epsilon \to 0 \implies U_k(\epsilon)$ is unitary 
    \item $U_k(\epsilon_1) U_k(\epsilon_2) = (I + {i \over \hbar} \epsilon_1 L_k) (I + {i \over \hbar} \epsilon_2 L_k) = I + {i \over \hbar} (\epsilon_1 + \epsilon_2) L_k + O(\epsilon^2)$ as expected
  \end{itemize}
  Hence, $U_k(\epsilon) = I + {i \over \hbar} \epsilon L_k$ is an appropriate space rotation transformation for Hilbert space kets $\ket{\psi}$
  \begin{itemize}
    \item $U_k(\epsilon) \ket{\psi} = \psi(T^{-1}_k(\epsilon) x)$ since space is rotated, the inverse mapping preserves $\psi(x)$ values
    \item For $U_3(\epsilon)$: $U_3(\epsilon) \psi(x) = \psi(T^{-1}_3(\epsilon) x) = \psi(x_1 + \epsilon x_2,x_2 - \epsilon x_1,x_3) = \psi(x) + \epsilon x_2 {\partial \psi \over \partial x_1} - \epsilon x_1 {\partial \psi \over \partial x_2} + O(\epsilon^2)$
    \item As $\epsilon \to 0$, $U_3(\epsilon) \psi(x) \to \psi(x_1,x_2,x_3) + \epsilon (x_2 {\partial \psi \over \partial x_1} - x_1 {\partial \psi \over \partial x_2}) = [I + \epsilon (x_2 {\partial \over \partial x_1} - x_1 {\partial \over \partial x_2}) ] \psi(x)$
    \item Generally, $U_k(\epsilon) \psi(x) = [I + \epsilon \epsilon_{ijk} x_i {\partial \over \partial x_j}] \psi(x) \implies L_k = -i \hbar \epsilon_{ijk} x_i {\partial \over \partial x_j}$
    \item The operators $L_k = -i \hbar \epsilon_{ijk} x_i {\partial \over \partial x_j}$ are called the orbital angular momentum operators
  \end{itemize}
\end{frame}

\begin{frame}
  \frametitle{Orbital Angular Momentum Operators II}
  \vspace{1mm}
  The orbital angular momentum operators $L_k = -i \hbar \epsilon_{ijk} x_i {\partial \over \partial x_j}$ is the generator of infinitesimal rotations in space
  \begin{itemize}
    \item For finite rotations by angle $\theta$: $R_k(\theta) = lim_{n \to \infty} [U_{k}({\theta \over n})]^n = lim_{n \to \infty} [I + {i \over \hbar} {\theta \over n} L_k]^n = \exp({i \over \hbar} \theta L_k)$
  \end{itemize}
  The commutator of any pair of the orbital angular momentum operators
  \begin{itemize}
    \item $[L_i,L_j] = [-i \hbar \epsilon_{mni} x_m {\partial \over \partial x_n},-i \hbar \epsilon_{pqj} x_p {\partial \over \partial x_q}] = -i \hbar \epsilon_{mni} \epsilon_{pqj} (x_m {\partial \over \partial x_n} x_p {\partial \over \partial x_q} - x_p {\partial \over \partial x_q} x_m {\partial \over \partial x_n})$
    \item $[L_i,L_j] = -i \hbar \epsilon_{mni} \epsilon_{pqj} (x_m (\delta_{pn} {\partial \over \partial x_q} + x_p {\partial^2 \over \partial x_n x_q}) - x_p (\delta_{mq} {\partial \over \partial x_n} + x_m {\partial^2 \over \partial x_n x_q}))$
    \item $[L_i,L_j] = -i \hbar \epsilon_{mni} \epsilon_{pqj} (x_m (\delta_{pn} {\partial \over \partial x_q}) - x_p (\delta_{mq} {\partial \over \partial x_n})) = -i \hbar (\epsilon_{mni} \epsilon_{nqj} x_m {\partial \over \partial x_q} - \epsilon_{qni} \epsilon_{pqj} x_p {\partial \over \partial x_n})$
    \item $[L_i,L_j] = -i \hbar (x_j {\partial \over \partial x_i} - x_i {\partial \over \partial x_j}) = i \hbar (x_i {\partial \over \partial x_j} - x_j {\partial \over \partial x_i}) = i \hbar \epsilon_{ijk} L_k$
  \end{itemize}
  Hence, $L_i$ belong to the family of case III operators
  \begin{itemize}
    \item The operators $L^2 = L_i L_i$ and $L_i$ have discrete eigenvalues
    \item The eigenvalues of $L^2$ are 
    \item The eigenvalues of $L_i$ are
  \end{itemize}
\end{frame}

\begin{frame}
  \frametitle{Internal Degrees of Freedom}
  \vspace{1mm}
  If the system particles have an extra set $s_i$ of degrees of freedom independent of the positions $x_i$
  \begin{itemize}
    \item The kets $\ket{\psi_x}$ describing the particles states as functions of $x_i$ are vectors of a Hilbert space $H_x$
    \item The kets $\ket{\psi_s}$ describing the particles states as functions of $s_i$ are vectors of a different Hilbert space $H_s$
    \item The kets $\ket{\psi}$ describing the particles states are kets in the product space $H = H_x \otimes H_s$
    \item The kets $\ket{\psi_s}$ are generally linear combinations of the basis kets $\ket{i_x}$ of $H_x$ and $\ket{j_s}$ of $H_s$
    \item Operators acting on $H$ vectors are the tensor products of operators acting on $H_x$ and ones acting on $H_s$
    \item $H_x$ operators do not affect $\ket{i_x}$ and $H_s$ operators do not affect $\ket{j_s}$
    \item $H_x$ operators commite with $H_s$ operators
  \end{itemize}
\end{frame}

\begin{frame}
  \frametitle{Spherical Polar Coordinates}
  \vspace{1mm}
  In spherical polar coordinates $(r,\theta,\phi)$ are related to Cartesian coordinates by
  \begin{itemize}
    \item $r = \sqrt{x^2 + y^2 + z^2}$, $\theta = \tan{y \over x}$ and $\phi = \tan{x^2 + y^2 \over z^2}$
  \end{itemize}
  The derivatives of the spherical coordinates with respect to Cartesian coordinates become
  \begin{itemize}
    \item ${\partial r \over \partial x} = \cos{\theta} \sin{\phi}$, ${\partial r \over \partial y} = \sin{\theta} \sin{\phi}$ and ${\partial r \over \partial z} = \cos{\phi}$
    \item ${\partial \theta \over \partial x} = -{\sin{\theta} \over r \sin{\phi}}$, ${\partial \theta \over \partial y} = {\cos{\theta} \over r \sin{\phi}}$ and ${\partial \theta \over \partial z} = 0$
    \item ${\partial \phi \over \partial x} = {\cos{\theta} \cos{\phi} \over r}$, ${\partial \phi \over \partial y} = {\sin{\theta} \cos{\phi} \over r}$ and ${\partial \phi \over \partial z} = -{\sin{\phi} \over r}$
  \end{itemize}
  A position vector $\vec{r} = x \hat{x} + y \hat{y} + z \hat{z} = r \cos{\theta} \sin{\phi} \hat{x} + r \sin{\theta} \sin{\phi} \hat{y} + r \cos{\phi} \hat{z}$ for Cartesian unit vectors $\hat{x},\hat{y},\hat{z}$
  \begin{itemize}
    \item $||{\partial \vec{r} \over \partial r}|| \hat{r} = {\partial \vec{r} \over \partial r} = \cos{\theta} \sin{\phi} \hat{x} + \sin{\theta} \sin{\phi} \hat{y} + \cos{\phi} \hat{z} \implies \hat{r} = \cos{\theta} \sin{\phi} \hat{x} + \sin{\theta} \sin{\phi} \hat{y} + \cos{\phi} \hat{z}$
    \item $||{\partial \vec{r} \over \partial \theta}|| \hat{\theta} = {\partial \vec{r} \over \partial \theta} = -r \sin{\theta} \sin{\phi} \hat{x} + r \cos{\theta} \sin{\phi} \hat{y} \implies \hat{\theta} = -\sin{\theta} \hat{x} + \cos{\theta} \hat{y}$
    \item $||{\partial \vec{r} \over \partial \phi}|| \hat{\phi} = {\partial \vec{r} \over \partial \phi} = r \cos{\theta} \cos{\phi} \hat{x} + r \sin{\theta} \cos{\phi} \hat{y} - r \sin{\phi} \hat{z} \implies \hat{\phi} = \cos{\theta} \cos{\phi} \hat{x} + \sin{\theta} \cos{\phi} \hat{y} - \sin{\phi} \hat{z}$
  \end{itemize}
  By inversion, the unit vectors in Cartesian coordinates in terms of those in spherical coordinates become
  \begin{itemize}
    \item $\hat{x} = \cos{\theta} \sin{\phi} \hat{r} - \sin{\theta} \hat{\theta} + \cos{\theta} \cos{\phi} \hat{\phi}$ , $\hat{y} = \sin{\theta} \sin{\phi} \hat{r} + \cos{\theta} \hat{\theta} + \sin{\theta} \cos{\phi} \hat{\phi}$
    \item $\hat{z} = \cos{\phi} \hat{r} - \sin{\phi} \hat{\phi}$
  \end{itemize}
\end{frame}

\begin{frame}
  \frametitle{Spherical Polar Coordinates II}
  \vspace{1mm}
  The gradient operator $\nabla = \hat{x} {\partial \over \partial x} + \hat{y} {\partial \over \partial y} + \hat{z} {\partial \over \partial z}$, in spherical coordinates
  \begin{itemize}
    \item $\hat{x} {\partial \over \partial x} = (\cos{\theta} \sin{\phi} \hat{r} - \sin{\theta} \hat{\theta} + \cos{\theta} \cos{\phi} \hat{\phi})(\cos{\theta} \sin{\phi} {\partial \over \partial r} - {\sin{\theta} \over r \sin{\phi}} {\partial \over \partial \theta} + {\cos{\theta} \cos{\phi} \over r} {\partial \over \partial \phi})$
    \item $\hat{y} {\partial \over \partial y} = (\sin{\theta} \sin{\phi} \hat{r} + \cos{\theta} \hat{\theta} + \sin{\theta} \cos{\phi} \hat{\phi})(\sin{\theta} \sin{\phi} {\partial \over \partial r} + {\cos{\theta} \over r \sin{\phi}} {\partial \over \partial \theta} + {\sin{\theta} \cos{\phi} \over r} {\partial \over \partial \phi})$
    \item $\hat{z} {\partial \over \partial z} = (\cos{\phi} \hat{r} - \sin{\phi} \hat{\phi})(\cos{\phi} {\partial \over \partial r} - {\sin{\phi} \over r} {\partial \over \partial \phi})$
  \end{itemize}
  By summing all terms $\nabla = \hat{r} {\partial \over \partial r} + \hat{\theta} {1 \over r \sin{\phi}} {\partial \over \partial \theta} + \hat{\phi} {1 \over r}{\partial \over \partial \phi}$
\end{frame}

\begin{frame}
  \frametitle{Angular Momentum Transformation}
  \vspace{1mm}
  The angular momentum vector in CM is defined to be $L=r \times p \implies L_k=\epsilon_{ijk} r_i p_j$
  \begin{itemize}
  	\item Where $r_i$ and $p_i$ are the position and momentum vectors respectively
  \end{itemize}
  One can construct an analogous operator $L_k$ in QM for angular momentum
  \begin{itemize}
  	\item $L_k = \epsilon_{ijk} X_i P_j$ with $X_i$ and $P_i$ being the position and momentum operators respectively 
  	\item Like $X_k$ and $P_k$, $L_k$ is a vector operator 
  \end{itemize}
  Construct the operator $T(\epsilon_k) = I - {i \over \hbar} \epsilon_k L_k$ for the infinitesimal real vector $\epsilon_k$
  \begin{itemize} 
  	\item $T(\epsilon_k)$ is an infinitesimal transformation
  	\item $T(\epsilon_k) \ket{\psi} = \ket{\psi} - {i \over \hbar} \epsilon_k \epsilon_{ijk} X_i P_j \ket{\psi}$
  	\item $T(\epsilon_k) \ket{\psi} = \psi(x_1,x_2,x_3) - {i \over \hbar} [\epsilon_1 (X_2 P_3 - X_3 P_2) + \epsilon_2 (X_3 P_1 - X_1 P_3) + \epsilon_3 (X_1 P_2 - X_2 P_1)] \psi(x_1,x_2,x_3)$
  	\item $T(\epsilon_k) \ket{\psi} = \psi(x_1,x_2,x_3) - \epsilon_1 (x_2 \psi_{,3} - x_3 \psi_{,2}) - \epsilon_2 (x_3 \psi_{,1} - x_1 \psi_{,3}) - \epsilon_3 (x_1 \psi_{,2} - x_2 \psi_{,1})$
  	\item $T(\epsilon_k) \ket{\psi} = \psi(x_1,x_2,x_3) + \psi_{,1} (\epsilon_3 x_2 - \epsilon_2 x_3) + \psi_{,2} (\epsilon_1 x_3 - \epsilon_3 x_1) + \psi_{,3} (\epsilon_2 x_1 - \epsilon_1 x_2)$
  	\item $T(\epsilon_k) \ket{\psi} = \psi(x_1 + \epsilon_3 x_2 - \epsilon_2 x_3,x_2 + \epsilon_1 x_3 - \epsilon_3 x_1,x_3 + \epsilon_2 x_1 - \epsilon_1 x_2) - O(\epsilon^2)$
  \end{itemize}
  Hence, $T(\epsilon_k)$ is the infinitesimal rotation transformation in 3-dimensional space
\end{frame}

\begin{frame}
  \frametitle{Angular Momentum Operator}
  \vspace{1mm}
  Angular momentum operator $L_k = \epsilon_{ijk} X_i P_j$ is the generator of infinitesimal rotations in space
  \begin{itemize}
  	\item In position space: $L_k = \epsilon_{ijk} X_i P_j = {-i \over \hbar}\epsilon_{ijk} x_i {\partial \over \partial x_j}$
  	\item In momentum space: $L_k = \epsilon_{ijk} X_i P_j = {i \over \hbar}\epsilon_{ijk} p_j {\partial \over \partial p_i} = {-i \over \hbar}\epsilon_{jik} p_j {\partial \over \partial p_i} = {-i \over \hbar}\epsilon_{ijk} p_i {\partial \over \partial p_j}$
  \end{itemize}
  The operator is symmetric in both position and momentum spaces $\implies$ the transformation $T(\epsilon_k) = I - {i \over \hbar} \epsilon_k L_k$ is also symmetric in both spaces
  \begin{itemize}
    \item $T(\epsilon_k) \ket{\psi} = \psi(p_1 + \epsilon_3 p_2 - \epsilon_2 p_3,p_2 + \epsilon_1 p_3 - \epsilon_3 p_1,p_3 + \epsilon_2 p_1 - \epsilon_1 p_2) - O(\epsilon^2)$
  	\item The application of $T(\epsilon_k)$ to wave functions in either spaces transforms them in the same way
  	\item This means that $L_k$ is the infinitesimal rotation generator in momentum space as well
    \item Since both position and momentum vectors transform the same way upon space rotations, the wave functions in both spaces must transform the same way for consistency $\implies L_k$ is a consistent angular momentum operator
    \item This would not be true if the linear momentum operator was chosen to be anything different than $P_k = -i \hbar {\partial \over \partial x_k}$
  \end{itemize}
\end{frame}

\begin{frame}
  \frametitle{Angular Momentum Operator II}
  \vspace{1mm}
  QM Angular momentum operator properly describes rotations in space

  The construction of $L_k$ was guided only by analogy to CM angular momentum vector

  Another way to construct $L_k$ is to impose space rotations analogous to CM on coordinates and momenta directly
    \begin{itemize}
    \item Position expectation for a wave function $\ket{\psi}$ is $\expval{X_k} = \expval{X_k}{\psi}$
    \item After rotation, the expected rotated position $\expval{X_k}_{R} = \expval{X_k}{\psi_{R}} = \expval{T^{\dagger}(\epsilon_j) X_k T(\epsilon_j)}{\psi}$
    \item The rotated position operator becomes $\tilde{X}_{k} = T^{\dagger}(\epsilon_j) X_k T(\epsilon_j)$
    \item For infinitesimal rotations, and analogous to CM $\expval{X_k}_{R} = \expval{X_k} + \epsilon_{ijk} \epsilon_{i} x_{j}$
    \item $\expval{X_k}_{R} = \expval{T^{\dagger}(\epsilon_j) X_k T(\epsilon_j)} = \expval{[I + {i \over \hbar} \epsilon_n L_n] X_k [I - {i \over \hbar} \epsilon_m L_m]}  = \expval{X_k} + {-i \over \hbar} \epsilon_n \expval{X_k L_n - L_n X_k}$
    \item Hence, $\epsilon_i \expval{[X_k,L_i]} = i \hbar \epsilon_{ijk} \epsilon_{i} \expval{X_{j}} \implies [X_k,L_i] = i \hbar \epsilon_{ijk} X_{j}$ since $\epsilon_{i}$ is arbitrary
    \item Similar reasoning can be applied to linear momentum rotation and operator to conclude that $[P_k,L_i] = i \hbar \epsilon_{ijk} P_{j}$
    \item The above relationships are uncertainty relationships between positions/linear momenta and angular momenta
  \end{itemize}
\end{frame}

\begin{frame}
  \frametitle{Angular Momentum Operator III}
  \vspace{1mm}
  The choice of $L_k = \epsilon_{ijk} X_i P_j$ satisfies the angular momentum uncertainty relationships $\implies L_k$ is a proper angular momentum operator 
  \begin{itemize}
    \item $[X_n,L_k] = [X_n,\epsilon_{ijk} X_i P_j] = \epsilon_{ijk} X_n X_i P_j - \epsilon_{ijk} X_i P_j X_n = \epsilon_{ijk} X_i [X_n,P_j] = i \hbar \epsilon_{ijk} X_i \delta_{n,j} = i \hbar \epsilon_{ink} X_i = i \hbar \epsilon_{kjn} X_j$
    \item $[P_n,L_k] = [P_n,\epsilon_{ijk} X_i P_j] = \epsilon_{ijk} P_n X_i P_j - \epsilon_{ijk} X_i P_j P_n = \epsilon_{ijk} [P_i,X_n] P_j = -i \hbar \epsilon_{ijk} \delta_{i,n} P_j = -i \hbar \epsilon_{njk} P_j = i \hbar \epsilon_{kjn} P_j$
    \item Consequently, $[L_i,X_j X_j] = [L_i,P_j P_j] = 0$ for all $i$
  \end{itemize}
  The exact form of $L_k$ can be derived from the uncertainty relationships

  For finite rotations by angles $\theta_k$ about the $3$ axes, the rotation operator $U(theta_k)$
  \begin{itemize}
    \item $U(\theta_k) = lim_{N \to \infty} (T({\theta_k \over N}))^{N} = lim_{N \to \infty}(I - {i \over \hbar} {\theta_k \over N} L_k)^N = exp({-i \over \hbar} \theta_k L_k)$
  \end{itemize}
  By going to cylindrical polar coordinates in a plane, the rotation operator about the axis normal to the plane becomes $L_n = -i \hbar {\partial \over \partial \phi_n}$ where $\phi_n$ is the angle in the plane
  \begin{itemize}
    \item Can be derived by considering $L_3$ in $x_1-x_2$ plane with $\phi$ being the angle between a vector in the plane and the positive direction of the $x_1$ axis
    \item Same reasoning applies to $L_1$, $L_2$ or any angular momentum operator rotating in any plane
    
  \end{itemize}
\end{frame}

\begin{frame}
  \frametitle{Angular Momentum Operator IV}
  \vspace{1mm}
  To prove $L_n = -i \hbar {\partial \over \partial \phi_n}$, consider $L_3$ in $x_1 - x_2$ plane
  \begin{itemize}
    \item $L_3 = X_1 P_2 - P_1 X_2$, using cylindrical polar coordinates $r,\phi$
    \item $X_1 = x_1 = r cos(\phi)$, $X_2 = x_2 = r sin(\phi)$
    \item $P_1 = -i \hbar {\partial \over \partial x_1} = -i \hbar [cos(\phi) {\partial \over \partial r} - {sin(\phi) \over r} {\partial \over \partial \theta}]$
    \item $P_2 = -i \hbar {\partial \over \partial x_1} = -i \hbar [sin(\phi) {\partial \over \partial r} + {cos(\phi) \over r} {\partial \over \partial \theta}]$
    \item $L_3 = -i \hbar [r cos(\phi) (sin(\phi) {\partial \over \partial r} + {cos(\phi) \over r} {\partial \over \partial \theta}) - r sin(\phi) (cos(\phi) {\partial \over \partial r} - {sin(\phi) \over r} {\partial \over \partial \theta})] = -i \hbar {\partial \over \partial \phi}$
  \end{itemize}
  For finite rotations $U(\theta_k) = exp(-{i \over \hbar} \theta_k L_k) = exp(-{i \over \hbar} \theta_k (-i \hbar {\partial \over \partial \phi_k})) = exp(-\theta_k {\partial \over \partial \phi_k})$
  \begin{itemize}
    \item $U(\theta) \psi(r,\phi) = exp(-\theta {\partial \over \partial \phi}) \psi(r,\theta) = \psi(r,\theta) - \theta \psi_{,\phi} - {\theta^2 \over 2} \psi_{,\phi \phi} + ... = \phi(r,\phi - \theta)$
    \item The correct finite rotation is reproduced from the finite rotation operator
    \item It follows that $U(\theta_1) U(\theta_2) = U(\theta_1 + \theta_2)$ as expected
  \end{itemize}
\end{frame}

\begin{frame}
  \frametitle{Angular Momentum Operator V}
  \vspace{1mm}
  For the angular momentum vector $\textbf{L} = L_k$
  \begin{itemize}
    \item $\textbf{L} \times \textbf{L} = \epsilon_{ijk} L_i L_j = \epsilon_{ijk} \epsilon_{mni} X_m P_n \epsilon_{stj} X_s P_t = (\delta_{jm} \delta_{kn} - \delta_{jn} \delta_{km}) \epsilon_{stj} X_m P_n X_s P_t$
    \item $\epsilon_{ijk} L_i L_j = \epsilon_{stm} (X_m P_k - X_k P_m) X_s P_t = (X_m P_k - X_k P_m) L_m = i \hbar \delta_{km} L_m = i \hbar L_k$
  \end{itemize}
  For two angular momentum operators $L_i$ and $L_j$
  \begin{itemize}
    \item $[L_i,L_j] = [\epsilon_{mni} X_m P_n,L_j] = \epsilon_{mni} (X_m [P_n,L_j] + [X_m,L_j] P_n)$
    \item $[L_i,L_j] = i \hbar \epsilon_{mni} (\epsilon_{jsn} X_m P_s + \epsilon_{jtm} X_t P_n) = i \hbar (\delta_{ij} X_m P_m - X_j P_i + X_i P_j - \delta_{ij} X_n P_n)$
    \item $[L_i,L_j] = i \hbar (X_i P_j - X_j P_i) = i \hbar \epsilon_{ijk} L_k$
    \item Angular momenta between different directions do not commute, unlike positions and linear momenta
  \end{itemize}
  The squared angular momentum operator $L^2 = L_i L_i$
  \begin{itemize}
    \item $[L^2,L_i] = [L_j L_j,L_i] = L_j [L_j,L_i] + [L_j,L_i] L_j = i \hbar \epsilon_{jik} L_j L_k + i \hbar \epsilon_{jik} L_k L_j$
    \item $[L^2,L_i] = i \hbar (\epsilon_{jik} L_j L_k - \epsilon_{jik} L_j L_k) = 0$ for all $i$
  \end{itemize}
\end{frame}

\begin{frame}
  \frametitle{Rotational Invariance}
  \vspace{1mm}
  If the system Hamiltonian commutes with $L_k \implies k$ direction angular momentum is conserved 
  \begin{itemize}
    \item $[H,L_k] = 0 \implies \dot{\expval{L_k}} = 0 \implies L_k$ is conserved
    \item There is a common eigenbasis for both $H$ and $L_k$ operators
    \item Starting from an energy eigenstate, the system's state remains in this eigenstate
    \item The energy eigenstate is also an angular momentum eigenstate
    \item The system remains in the angular momentum eigenstate $\implies$ angular momentum is conserved
  \end{itemize}
  Systems that are rotationally invariant 
  \begin{itemize}
    \item Guarantee space isotropy (systems evolve the same way before and after rotation)
    \item $U^{\dagger}HU = H$ where $U$ is the rotation operator (finite or infinitesimal) 
    \item $[H,L_i] = 0$ for all $i \implies [H,L_i L_i] = [H,L^2] = 0 \implies$ all $L_i$ and $L^2$ are conserved
    \item However, there is no common basis for $H$, $L^2$ and all $L_i$ simultanously since $[L_i,L_j] \neq 0$
    \item There is a common basis for $H$ and each $L_i$ individually, but not all $3$ together
    \item Usually a common basis for $H$, $L^2$ and $L_3$ is used
  \end{itemize}
\end{frame}

\begin{frame}
  \frametitle{Angular Momentum Eigenvalues}
  \vspace{1mm}
  The rotational energy operator is $L^2$, it commutes with all $L_i$

  To solve for the common eigenkets of $L^2$ and $L_3$ (analysis is similar for $L_1$ and $L_2$)
  \begin{itemize}
    \item Define the operators $L_{+} = L_1 + i L_2$ and $L_{-} = L_1 - i L_2 \implies [L_3,L_{+}] = \hbar L_{+}$ and $[L_3,L_{-}] = -\hbar L_{-}$
    \item Since $L^2$ commutes with all $L_k \implies [L^2,L_{+}] = [L^2,L_{-}] = 0$
  \end{itemize}
  Let $\ket{\alpha \beta}$ be a common eigenket for $L^2$ and $L_3$ with eigenvalues $\alpha$ and $\beta$ respectively 
  \begin{itemize}
    \item $L_3 L_{+} \ket{\alpha \beta} = (L_{+} L_3 + \hbar L_{+}) \ket{\alpha \beta} = (L_{+} \beta + \hbar L_{+}) \psi_{Em} = (\beta + \hbar) L_{+} \ket{\alpha \beta}$
    \item Hence if $\ket{\alpha \beta}$ is an $L_3$ eigenket, then $L_{+} \ket{\alpha \beta}$ is also an $L_3$ eigenket
    \item The effect of $L_{+}$ application is to transform the eigenket to the one with an angular momentum that is $\hbar$ higher $\implies L_{+}$ is a raising operator
    \item Similarly, $L_3 L_{-} \ket{\alpha \beta} = (L_{-} L_3 - \hbar L_{-}) \ket{\alpha \beta} = (L_{-} \beta + \hbar L_{-}) \ket{\alpha \beta} = (\beta - \hbar) L_{-} \ket{\alpha \beta}$
    \item The application of $L_{-}$ transforms the eigenket to the one with an angular momentum that is $\hbar$ lower $\implies L_{-}$ is a lowering operator
    \item $L^2 L_{+} \ket{\alpha \beta} = L_{+} L^2 \ket{\alpha \beta} = \alpha L_{+} \ket{\alpha \beta}$ and $L^2 L_{-} \ket{\alpha \beta} = L_{-} L^2 \ket{\alpha \beta} = \alpha L_{-} \ket{\alpha \beta}$
    \item The application of $L_{+}$ or $L_{-}$ on $L^2$ eigenkets does not change them
  \end{itemize}
\end{frame}

\begin{frame}
  \frametitle{Angular Momentum Eigenvalues II}
  \vspace{1mm}
  Angular momentum raising/lowering operators increase/decrease angular momentum indefinitely

  However, for a given system energy, the angular momentum cannot be arbitrarily large in either direction
  \begin{itemize}
    \item From CM, the angular momentum can at most be $l_{3}^{2} < l^2$ where $l^2$ is the rotational kinetic energy
    \item In QM, $\expval{L^2 - L_{3}^{2}}{\psi} = \expval{L_{1}^2 + L_{2}^{2}}{\psi} \geq 0$ since $\expval{L_{i}^2}{\psi}$ \geq 0$ for all $\psi$
    \item Hence, $\expval{L^2 - L_{3}^{2}}{\alpha \beta} = \alpha - \beta^2 \geq 0 \implies \alpha \geq \beta^2$
  \end{itemize}
  Hence, there are states $\ket{\alpha \beta_{max}}$ and $\ket{\alpha \beta_{min}}$ that cannot be further raised/lowered
  \begin{itemize}
    \item $L_{+} \ket{\alpha \beta_{max}} = L_{-} \ket{\alpha \beta_{min}} = 0 \implies L_{-} L_{+} \ket{\alpha \beta_{max}} = L_{+} L_{-} \ket{\alpha \beta_{min}} = 0$
    \item $L_{-} L_{+} = L_{1}^{2} + i L_1 L_2 - i L_2 L_1 + L_{2}^{2} = L^{2} - L_{3}^{2} + i[L_1,L_2] = L^{2} - L_{3}^{2} - \hbar L_3$ 
    \item $L_{+} L_{-} = L_{1}^{2} - i L_1 L_2 + i L_2 L_1 + L_{2}^{2} = L^{2} - L_{3}^{2} - i[L_1,L_2] = L^{2} - L_{3}^{2} + \hbar L_3$
    \item $L_{-} L_{+} \ket{\alpha \beta_{max}} = (L^{2} - L_{3}^{2} - \hbar L_3,L_2) \ket{\alpha \beta_{max}} = (\alpha - \beta_{max}^{2} - \hbar \beta_{max}) \ket{\alpha \beta_{max}} = 0$
    \item Hence, $\alpha = \beta_{max} (\beta_{max} + \hbar)$ and similarly $\alpha = \beta_{min} (\beta_{min} - \hbar)$
    \item This is only true if $\beta_{max} = -\beta_{min} \implies \beta_{max} - \beta_{min} = 2 \beta_{max}$
    \item Between $\beta_{max}$ and $\beta_{min}$, there are $N$ raising/lowering steps $\implies \beta_{max} = \beta_{min} + N \hbar$
    \item From $\beta_{max}$ and $\beta_{min}$ symmetry, $\beta_{max} - \beta_{min} = 2 \beta_{max} = N \hbar \implies \beta_{max} = -\beta_{min} = {N \over 2} \hbar$
  \end{itemize}
\end{frame}

\begin{frame}
  \frametitle{Angular Momentum Eigenvalues III}
  \vspace{1mm}
  The angular momentum $L_3$ eigenvalues lie in the interval $[\beta_{min},\beta_{max}]$
  \begin{itemize}
    \item Let $l_3 = m \hbar \implies m \in \{-{N \over 2},-{N \over 2} + 1, ... ,{N \over 2} - 1,{N \over 2}\} \implies m$ takes $N + 1$ values
    \item For even $N$, $m$ takes integer values while for odd $N$, it takes half-integer values
    \item $m$ is called the magnetic quantum number of the particle
    \item $N$ determines the total rotational energy of the system in all $3$ directions
    \item $N$ and $m$ determine the rotational state of the rotation common eigenkets
    \item Angular momentum quantization emerges without resorting to any system $\implies$ it is general for all wave functions
    \item Since $\alpha = \beta_{max} (\beta_{max} + \hbar)$ and $\beta_{max} = {N \over 2} \hbar \implies \alpha = \hbar^2 l (l + 1)$ where $l = {N \over 2}$
    \item $N$ fixes $l = {N \over 2}$ and $m \in \{-l,-l + 1, ... ,l - 1,l\}$
  \end{itemize}
  Given $N$ (or $l$) and $m$, the eigenvalues of $L^2$ and $L_3$ are $\hbar^2 l (l + 1)$ and $m \hbar$ respectively
  \begin{itemize}
    \item Note that $m$ does not affect $L^2$, it only partitions some of $L^2$ energy to $L_3$
    \item $N$ however is a measure of the rotational kinetic energy of the particle
  \end{itemize}
\end{frame}

\begin{frame}
  \frametitle{Angular Momentum Eigenvalues IV}
  \vspace{1mm}
  The $L_{3}^2$ eigenvalue $l_{3}^{2}$ is $L_3 L_3 \ket{\alpha \beta} = \beta^2 \ket{\alpha \beta} \implies l_{3}^{2} = \hbar^2 m^2$
  \begin{itemize}
    \item The maximum $l_{3}^{2} = \hbar^2 m_{max}^{2} = \hbar^2 l^2 < \hbar^2 l(l + 1) = L^2$ eigenvalue
    \item $L_{3}^{2}$ eigenvalue $< L^2$ eigenvalue for all $l$ except for $l = 0$
    \item Unless the rotational kinetic energy (RKE) is zero, the RKE in any direction cannot be equal to the total RKE
  \end{itemize}
    The particle cannot rotate about one axis exclusively, it has to always rotate around axes that have components in more than one direction !!!
  \begin{itemize}
    \item This is unintuitive but true, particle's rotation cannot be aligned with any direction
    \item Because angular momentum operators do not commute $\implies$ particle does not have definite angular momenta in all directions simultaneously
    \item A particle having a defined angular momentum in one direction will have a spread angular momentum in other directions
    \item This spread is accounted for by allocating some of the rotational kinetic energies to other directions
    \item Hence, some of this energy is inaccessible to any single rotation direction 
  \end{itemize}
\end{frame}

\begin{frame}
  \frametitle{Angular Momentum Eigenkets}
  \vspace{1mm}
  The angular momentum operator $L_k = \epsilon_{ijk} X_i P_j = -i \hbar \epsilon_{ijk} x_i {\partial \over \partial x_j} \implies \textbf{L} = {\hbar \over i} \vec{r} \times \nabla$
  
  In spherical polar coordinates, $\vec{r} = r \hat{r}$ and $\nabla = \hat{r} {\partial \over \partial r} + \hat{\theta} {1 \over r \sin{\phi}} {\partial \over \partial \theta} + \hat{\phi} {1 \over r}{\partial \over \partial \phi}$
  \begin{itemize}
    \item $\textbf{L} = {\hbar \over i} r \hat{r} \times (\hat{r} {\partial \over \partial r} + \hat{\theta} {1 \over r \sin{\phi}} {\partial \over \partial \theta} + \hat{\phi} {1 \over r}{\partial \over \partial \phi}) = {\hbar \over i} (\hat{\theta} {\partial \over \partial \phi} - \hat{\phi} {1 \over \sin{\phi}} {\partial \over \partial \theta})$
  \end{itemize}
  By resolving $\hat{\theta}$ and $\hat{\phi}$ in Cartesian coordinates
  \begin{itemize}
    \item $\textbf{L} = {\hbar \over i} ((-\sin{\theta} \hat{x} + \cos{\theta} \hat{y}) {\partial \over \partial \phi} - (\cos{\theta} \cos{\phi} \hat{x} + \sin{\theta} \cos{\phi} \hat{y} - \sin{\phi} \hat{z}) {1 \over \sin{\phi}} {\partial \over \partial \theta})$
    \item $L_x = i \hbar (\sin{\theta} {\partial \over \partial \phi} + \cos{\theta} \cot{\phi} {\partial \over \partial \theta})$, $L_y = i \hbar (\sin{\theta} \cot{\phi} {\partial \over \partial \theta} -\cos{\theta} {\partial \over \partial \phi})$ and $L_z = {-i \hbar} {\partial \over \partial \theta}$
    \item $L_{+} = \hbar [(-\sin{\theta} + i\cos{\theta}) \cot{\phi} {\partial \over \partial \theta} + (\cos{\theta} + i \sin{\theta}) {\partial \over \partial \phi}] = \hbar \exp(i \theta) [{\partial \over \partial \phi} + i \cot{\phi} {\partial \over \partial \theta}]$
    \item $L_{-} = \hbar [(\sin{\theta} + i\cos{\theta}) \cot{\phi} {\partial \over \partial \theta} + (-\cos{\theta} + i \sin{\theta}) {\partial \over \partial \phi}] = -\hbar \exp(-i \theta) [{\partial \over \partial \phi} - i \cot{\phi} {\partial \over \partial \theta}]$
    \item $L_{+} L_{-} = $
  \end{itemize}
\end{frame}

\begin{frame}
  \frametitle{Angular Momentum Eigenvalue Problem III}
  \vspace{1mm}
  To find the eigenvalues and eigenkets of the angular momentum operator, solve $L_i \ket{l_i} = l_i \ket{l_i}$
  \begin{itemize}
    \item In a plane normal to the rotation generated by $L_i$, the angular momentum operator becomes $L_n = -i \hbar {\partial \over \partial \phi_n}$
    \item Use cylindrical polar coordinates $r,\phi$ in the rotation plane
    \item The eigenvalue problem becomes $L_i \ket{l_i} = -i \hbar {\partial \over \partial \phi} l(r,\phi) = l l(r,\phi)$
    \item The most general solution to this differential equation is $l(r,\phi) = R(r) exp({i \over \hbar} l \phi)$
    \item $R(r)$ is a normalizable function in $r$ such that $\int_{0}^{\infty} R(r) r dr$ converges
  \end{itemize}
  Since $L$ is Hermitian $\implies \mel{\psi_1}{L}{\psi_2} = \mel{\psi_2}{L}{\psi_1}^{*} \implies \int_{0}^{\infty} \int_{0}^{2 \pi} \psi_1^{*} L \psi_{2} r dr d\phi = [\int_{0}^{\infty} \int_{0}^{2 \pi} \psi_2^{*} L \psi_{1} r dr d\phi]^{*}$
  \begin{itemize}
    \item $S = \mel{\psi_1}{L}{\psi_2} = -i \hbar \int_{0}^{\infty} \int_{0}^{2 \pi} \psi_1^{*} \psi_{2,\phi} r dr d\phi = -i \hbar [\int_{0}^{\infty} \int_{0}^{2 \pi} (\psi_1^{*} \psi_{2})_{,\phi} r dr d\phi - \int_{0}^{\infty} \int_{0}^{2 \pi} \psi_1^{*}_{,\phi} \psi_{2} r dr d\phi]$
    \item $S = -i \hbar [\int_{0}^{\infty} [\psi_1^{*} \psi_{2}]_{0}^{2 \pi} r dr - \int_{0}^{\infty} \int_{0}^{2 \pi} \psi_{2} \psi_1^{*}_{,\phi} r dr d\phi]$
    \item Let $T_1 = -i \hbar \int_{0}^{\infty} [\psi_1^{*} \psi_{2}]_{0}^{2 \pi} r dr$ and $T_2 = i \hbar \int_{0}^{\infty} \int_{0}^{2 \pi} \psi_{2} \psi_1^{*}_{,\phi} r dr d\phi \implies S = T_1 + T_2$
    \item $T_{2}^{*} = -i \hbar \int_{0}^{\infty} \int_{0}^{2 \pi} \psi_{2}^{*} \psi_1_{,\phi} r dr d\phi = \mel{\psi_2}{L}{\psi_1}^{*}$
    \item Hermitian $L \implies S = T_2 \implies T_1 = 0 \implies \psi_1(r,0) = \psi_1(r,2 \pi)$ and $\psi_2(r,0) = \psi_2(r,2 \pi)$ for all $\psi_1$ and $\psi_2$
  \end{itemize}
\end{frame}

\begin{frame}
  \frametitle{Angular Momentum Eigenvalue Problem IV}
  \vspace{1mm}
  Applying operator Hermiticity condition on its eigenkets $l(r,0) = l(r,2 \pi)$ for all $L$ eigenkets
  \begin{itemize}
    \item $l(r,\phi) = R(r) exp({i \over \hbar} l \phi) \implies l(r,0) = R(r)$ and $l(r,2 \pi) = R(r) exp({i \over \hbar} l 2 \pi) \implies exp({i \over \hbar} l 2 \pi) = 1$
    \item $exp({i \over \hbar} l 2 \pi) = 1$ is only true if the eigenvalue $l$ is an integer multiple of $\hbar$
    \item The eigenvalues of the angular momentum operator are $l = m \hbar$
    \item The eigenkets are $l(r,\phi) = R(r) exp(i m \phi)$ where $m$ is an integer
    \item The integer $m$ is called the magnetic quantum number
    \item These are the eigenkets for a rotation about a specific axis, there are other eigenkets for rotations about other normal axes
  \end{itemize}
  The Hermiticiy of the angular momentum operator leads to quantization for all systems

  The angular part of the angular momentum operator eigenket is $\Phi_m(\phi) = (2 \pi)^{-1/2} exp(i m \phi)$
  \begin{itemize}
    \item $\int_{0}^{2 \pi} \Phi_m^{*} \Phi_n d\phi = {1 \over {2 \pi}} \int_{0}^{2 \pi} exp(i (n - m) \phi) d\phi = {1 \over {2 \pi}} \delta_{nm}$
  \end{itemize}
\end{frame}

\begin{frame}
  \frametitle{Angular Momentum Eigenvalue Problem V}
  \vspace{1mm}
  For rotationally invariant problems $[H,L] = 0$ for a rotation generator in a plane

  Time independent Schrodinger's equation in cylindrical polar coordinates of the plane is
  \begin{itemize}
    \item $[-{\hbar^2 \over 2 \M} ({\partial^2 \over \partial r^2} + {1 \over r} {\partial \over \partial r} + {1 \over r^2} {\partial^2 \over \partial \phi^2}) + V(r)] \psi(r,\phi) = E \psi(r,phi)$
    \item For an eigenket of the angular momentum operator $\psi_{Em}(r,\phi) = R_{Em}(r) \Phi_m(\phi)$
    \item $[-{\hbar^2 \over 2 \M} ({\partial^2 \over \partial r^2} + {1 \over r} {\partial \over \partial r}- {m^2 \over r^2} ) + V(r)] R_{Em}(r) = E R_{Em}(r)$
    \item This is the eigenvalue problem for the radial part of the wave function
    \item As the potential of a rotationally invariant system changes, the angular part of the angular momentum eigenkets remain the same, only the radial part changes
    \item The function $\psi_{Em}(r,\phi) = R_{Em}(r) \Phi_m(\phi)$ becomes a common eigenket for both the energy and the angular momentum operators
    \item Note that there is a family of eigenkets $\psi_{Em}(r,\phi)$ for every magnetic quantum number with each having a different energy level
    \item Similarly, there is a family of eigenkets $\psi_{Em}(r,\phi)$ for each energy level with each having a different magnetic quantum number
  \end{itemize}
\end{frame}

\end{document}
